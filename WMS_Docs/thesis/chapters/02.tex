% \begin{itemize}
% \item sformułowanie problemu
% \item osadzenie tematu w kontekście aktualnego stanu wiedzy (\english{state of the art}) o poruszanym problemie
% \item  studia literaturowe \cite{bib:artykul,bib:ksiazka,bib:konferencja,bib:internet} -  opis znanych rozwiązań (także opisanych naukowo, jeżeli problem jest poruszany w publikacjach naukowych), algorytmów, 
% \end{itemize}


% Wzory  
% \begin{align}
% y = \frac{\partial x}{\partial t}
% \end{align}
% jak i pojedyncze symbole $x$ i $y$  składa się w trybie matematycznym.


%%%%%%%%%%%%%%%%%%%%%%%%


\chapter{Analiza tematu}

\section{Kontrola czasu pracy w historii}

\note{W tej części można opisać jak wyglądała kontrola czasu pracy w przeszłości. Należy zaznaczyć, jakie metody były stosowane, jakie były ich wady i zalety. Można również opisać, jakie były konsekwencje nieprzestrzegania zasad kontroli czasu pracy.}

\section{Przegląd dostępnych sposobów kontroli czasu}

\subsection{Metody tradycyjne}

Wiele małych przedsiębiorstw nie potrzebuje wprowadzenia zaawansowanych systemów kontroli pracowników i wciąż korzysta z metod tradycyjnych. Polegają one w większości na ręcznym wypełnianiu papierowych dokumentów, które następnie wymagają przetworzenia. Przykłady takich metod wymieniono poniżej.

\begin{itemize}
    \item \textbf{Indywidualne karty czasu pracy} - pracownik zaznacza swoją obecność na kartce papieru, a następnie podpisuje ją. Zaletą tej metody jest jej prostota i niski koszt, jednakże jest ona mało efektywna i podatna na błędy. Dodatkowo wymagane jest przechowywanie dużej ilości papierowych dokumentów oraz ich archiwizacji.
    \item \textbf{Arkusz kalkulacyjny} - metoda polegająca na prowadzeniu arkusza kalkulacyjnego, w którym podobnie jak w przypadku kart czasu pracy zapisywane są godziny pracy pracowników oraz zadania jakie wykonali. Arkusze pozwalają na jednoczesną kontrolę obecności pracowników oraz na monitorowanie ich postępów w pracy. Niestety, brak automatyzacji procesu skutkuje koniecznością ręcznego wypełniania arkuszy, co zwiększa ryzyko popełnienia błędów. Podobnie jak w przypadku kart pracy, konieczne jest przechowywanie i archiwizacja dokumentów.
    \item \textbf{Harmonogram pracy} - pozwala przełożonym na ustalenie grafiku pracy pracowników. Jest on następnie przekazywany podwładnym, którzy muszą przestrzegać ustalonych godzin pracy. Ta metoda wymaga wykorzystania dodatkowych narzędzi, takich jak karty pracy. Największą wadą tej metody jest brak możliwości przekazania informacji o zmianach w grafiku w czasie rzeczywistym, co może prowadzić do nieporozumień i konfliktów.
\end{itemize}

Połączenie kilku tradycyjnych metod pozwala na skuteczną kontrolę czasu pracy, lecz bez wykorzystania systemów informatycznych, proces ten jest bardzo czasochłonny i podatny na błędy. Korzystanie z papierowych dokumentów jest obarczone bardzo dużym ryzykiem utraty danych, nieautoryzowanego dostępu oraz błędów ludzkich.

\subsection{Systemy zarządzania kapitałem ludzkim}

Wraz z rozwojem technologii informatycznych wiele firm zdecydowało się na wprowadzenie rozwiązań HCM (ang. \english{Human Capital Management}). Pozwalają one na pełną automatyzację procesów związanych z działaniem kadr i płac. Oferują szereg funkcji związanych z zarządzaniem czasem pracy, rekrutacją, szkoleniami, wynagrodzeniami oraz rozwojem pracowników. Ich główną częścią jest moduł \english{Workforce Management}. Zazwyczaj dostęp do systemu odbywa się poprzez aplikację internetową, co umożliwia łatwy dostęp z dowolnego miejsca na świecie. Przykłady dostępnych systemów HCM wymieniono poniżej.

\begin{itemize}
    \item \textbf{Oracle HCM Cloud}\cite{bib:OracleHCM} - kompletne rozwiązanie chmurowe firmy Oracle, które łączy w sobie funkcje zarządzania personelem, procesami kadrowymi, rekrutacyjnymi i płacowymi. Jest używany m. in. przez FUJIFILM, Deutsche Bahn, czy Fujitsu.
    \item \textbf{SAP SuccessFactors HCM}\cite{bib:SAPHCM} - rozwiązanie chmurowe firmy SAP, które oferuje szereg funkcji w zakresie HR (ang. \english{Human Resources}). Zawiera w sobie moduły do zarządzania procesami kadrowymi, rekrutacyjnymi, szkoleniowymi, płacowymi i analitycznymi. Jest używany m. in. przez Microsoft, Nestle, Allianz.
    \item \textbf{MintHCM}\cite{bib:MintHCM} - oprogramowanie firmy eVolpe oparte o otwartoźródłowe systemy CRM (ang. \english{Customer Relationship Management}). Oferuje szereg funkcji związanych z zarządzaniem personelem, takich jak: rekrutacja, szkolenia, oceny pracownicze, czy zarządzanie czasem pracy i urlopami.  Korzystają z niego m. in. Empik, Poczta Polska, czy Asseco.
\end{itemize}

Głównym powodem, dla którego firmy decydują się wdrożyć systemy HCM jest ich pozytywny wpływ na efektywność pracy, a co za tym idzie - zwiększenie przychodów. Dobrze zaprojektowany system, który pozwala na załatwienie wielu spraw formalnych oraz administracyjnych w jednym miejscu ułatwia pracownikom codzienną pracę, pozwala na szybsze reagowanie na zmiany w organizacji i daje jasny wgląd do danych dotyczących ich wydajności. Dla kadry, system umożliwia monitorowanie działań i wyników pracowników, co może przełożyć się na premie i awanse.

Często w rozwiązaniach HCM brakuje funkcji związanej z przyznawaniem dostępów oraz kontrolą wejść i wyjść pracowników z firmy. W takich przypadkach konieczne jest zintegrowanie systemu HCM z systemem kontroli dostępu, co zwiększa koszty i skomplikowanie systemu. Dodatkowo, systemy HCM są zazwyczaj dostępne jedynie w formie chmurowej, co może być problemem dla firm, które chcą mieć pełną kontrolę nad danymi swoich pracowników.

\subsection{Systemy kontroli dostępu}

Celem systemów kontroli dostępu jest zapewnienie bezpieczeństwa w firmie poprzez kontrolę wejść i wyjść pracowników oraz gości. Ich głównym zadaniem jest zapewnienie bezpieczeństwa pracownikom oraz ochrona mienia firmy. Pozwalają one na identyfikację osób przemieszczających się po budynku oraz na kontrolę dostępu do poszczególnych pomieszczeń. Zdarza się, że systemy są zintegrowane z alarmami oraz monitoringiem. Zazwyczaj spotyka się je w dużych firmach, w których kontrola dostępu jest kluczowym elementem bezpieczeństwa. Takie systemy dostarczają m. in. firmy:

\begin{itemize}
    \item \textbf{Satel} - polska firma zajmująca się produkcją systemów alarmowych, monitoringowych i kontroli dostępu, której rozwiązania opierają się o technologię RFID. Możliwe jest ich wdrożenie lokalne oraz rozproszone,
    \item \textbf{Avigilon} - firma zajmująca się produkcją systemów monitoringu i kontroli dostępu. Ich rozwiązania opierają się głównie na technologiach bezprzewodowych oraz pinpadach.
\end{itemize}

Systemy kontroli dostępu są zazwyczaj stosowane w firmach, w których bezpieczeństwo jest kluczowym elementem. Dla pracowników ich użytkowanie jest proste i intuicyjne, a dostęp do poszczególnych pomieszczeń jest szybki i wygodny. Niestety, systemy te nie oferują funkcji związanych z zarządzaniem personelem i czasem pracy. W takich przypadkach konieczne jest zintegrowanie systemu kontroli dostępu z systemem HCM, co zwiększa koszty i skomplikowanie systemu.

\subsection{Podsumowanie}

Analiza dziedziny pozwala na stwierdzenia, że istnieje zapotrzebowanie na system łączący w sobie funkcje zarządzania personelem, kontroli czasu pracy oraz kontroli dostępu. Obecne rozwiązania są albo nieefektywne i przestarzałe, albo nie zawierają w sobie wszystkich funkcjonalności. W związku z tym zaprojektowanie i wdrożenie nowego systemu, który pozwoli na automatyzację wymienionych procesów, może przynieść wymierne korzyści dla firm. Taki system pozwoli na zwiększenie efektywności pracy, bezpieczeństwa oraz ułatwi pracownikom codzienną pracę. Dodatkowo, wykluczy on koszty ponoszone na utrzymanie kilku systemów oraz zintegrowanie ich ze sobą.

\note{tu może być jeszcze dopisana dywagacja na temat zasadności wdrożenia systemu zarządzania personelem}

\section{Najważniejsze funkcjonalności systemu}

\note{W tej części pracy można opisać, jakie funkcjonalności powinien posiadać system zarządzania personelem. Należy zaznaczyć, jakie są one najważniejsze, a także jakie korzyści mogą przynieść firmie.}