\chapter{Wymagania i narzędzia}
\label{ch:wymagania-i-narzedzia}

% \begin{itemize}
% \item wymagania funkcjonalne i niefunkcjonalne
% \item przypadki użycia (diagramy UML) -- dla prac, w których mają zastosowanie
% \item opis narzędzi, metod eksperymentalnych, metod modelowania itp.
% \item metodyka pracy nad projektowaniem i implementacją -- dla prac, w których ma to zastosowanie
% \end{itemize}

\section{Założenia projektowe}

\note{W tej części pracy należy opisać, jakie założenia przyjęto podczas tworzenia systemu. Należy zaznaczyć, jakie funkcje systemu są najważniejsze, a które mogą być pominięte.}

\subsection{Wymagania funkcjonalne}

\subsection{Wymagania niefunkcjonalne}

\section{Projekt systemu}

\note{W tej części pracy należy opisać, jakie narzędzia i technologie zostały wykorzystane podczas tworzenia systemu. Należy zaznaczyć, dlaczego wybrano dane narzędzia, a nie inne.}

\subsection{Technologie}

\subsection{Narzędzia}

\subsection{Metodyka pracy}

\section{Przypadki użycia}

\note{W tej części pracy należy przedstawić diagramy UML przypadków użycia systemu oraz je opisać.}

