%%%%%%%%%%%%%%%%%%%%%%%%%%%%%%%%%%%%%%%%%
% Academic Title Page
% LaTeX Template
% Version 2.0 (17/7/17)
%
% This template was downloaded from:
% http://www.LaTeXTemplates.com
%
% Original author:
% WikiBooks (LaTeX - Title Creation) with modifications by:
% Vel (vel@latextemplates.com)
%
% License:
% CC BY-NC-SA 3.0 (http://creativecommons.org/licenses/by-nc-sa/3.0/)
% 
% Instructions for using this template:
% This title page is capable of being compiled as is. This is not useful for 
% including it in another document. To do this, you have two options: 
%
% 1) Copy/paste everything between \begin{document} and \end{document} 
% starting at \begin{titlepage} and paste this into another LaTeX file where you 
% want your title page.
% OR
% 2) Remove everything outside the \begin{titlepage} and \end{titlepage}, rename
% this file and move it to the same directory as the LaTeX file you wish to add it to. 
% Then add \input{./<new filename>.tex} to your LaTeX file where you want your
% title page.
%
%%%%%%%%%%%%%%%%%%%%%%%%%%%%%%%%%%%%%%%%%

%----------------------------------------------------------------------------------------
%	PACKAGES AND OTHER DOCUMENT CONFIGURATIONS
%----------------------------------------------------------------------------------------
\documentclass[11pt,a4paper,titlepage,hidelinks]{article}

\usepackage[utf8]{inputenc}
\usepackage[polish]{babel}
\usepackage[T1]{fontenc}
\usepackage{float}
\usepackage{hyperref}
\usepackage{amsmath}
\usepackage{hyperref}
\usepackage[margin=1in]{geometry}
\usepackage{enumitem}

\usepackage{mathpazo} % Use the Palatino font by default


\newcommand\myemptypage{
	\null
	\thispagestyle{empty}
	\newpage
}

\newcommand\pageCounterMinusOne{
	\addtocounter{page}{-1}
}

\title{\Huge{Projekt inżynierski}\\\LARGE{Workforce Management System}}
\author{Piotr Marcol}
\date{\today}


\begin{document}

\selectlanguage{polish}

\begin{titlepage} % Suppresses displaying the page number on the title page and the subsequent page counts as page 1

    \newcommand{\HRule}{\rule{\linewidth}{0.5mm}} % Defines a new command for horizontal lines, change thickness here

    \center % Centre everything on the page

    %------------------------------------------------
    %	Headings
    %------------------------------------------------

    \textsc{\LARGE Politechnika Śląska}\\[0.5cm] % Main heading such as the name of your university/college

    \textsc{\Large Wydział Automatyki, Elektroniki i Informatyki}\\[1.5cm] % Major heading such as course name

    \textsc{\Large Projekt inżynierski}\\[0.5cm] % Minor heading such as course title

    \textsc{\LARGE Słownik}\\[0.2cm] % Minor heading such as course title

    %------------------------------------------------
    %	Title
    %------------------------------------------------

    \HRule\\[0.4cm]

    { \LARGE System zarządzania personelem \\[0.2cm]
    Workforce management system }\\[0.2cm] % Title of your document

    \HRule\\[1.5cm]

    %------------------------------------------------
    %	Author(s)
    %------------------------------------------------

    \begin{minipage}{0.5\textwidth}
        \begin{flushleft}
            % \large
            Piotr Marcol\\
            Inf. SSI, ISMIP\\
            Rok akademicki 2024/2025
        \end{flushleft}
    \end{minipage}
    % ~
    \begin{minipage}{0.4\textwidth}
        \begin{flushright}
            % \large
            \textit{Promotor}\\
            dr inż. Marcin Połomski
        \end{flushright}
    \end{minipage}



    %------------------------------------------------
    %	Logo
    %------------------------------------------------

    % \vfill\vfill
    % \includegraphics[width=0.5\textwidth]{polslLogo.jpeg}\\[1cm] % Include a department/university logo - this will require the graphicx package

    %----------------------------------------------------------------------------------------

    %------------------------------------------------
    %	Date
    %------------------------------------------------

    \vfill\vfill\vfill % Position the date 3/4 down the remaining page

    {\large\today} % Date, change the \today to a set date if you want to be precise

    \vfill % Push the date up 1/4 of the remaining page

\end{titlepage}

\newpage

Niniejszy dokument zawiera wytłumaczenia pojęć używanych podczas pracy z Systemem zarządzania personelem (Workforce Management System) realizowanym w ramach projektu inżynierskiego.

\begin{itemize}
    \item \textbf{System} - całość oprogramowania i sprzętu komputerowego, które składają się na poprawnie działającą aplikację.
    \item \textbf{Workforce Management} - zbiór procesów i narzędzi, które mają na celu zoptymalizowanie zarządzania personelem w organizacji.
    \item \textbf{Czytnik kart} - urządzenie, które umożliwia rozpoczęcie procesu autoryzacji pracownika na podstawie karty zbliżeniowej.
    \item \textbf{Logowanie} - proces, w którym użytkownik podaje swoje dane uwierzytelniające w celu dostępu do systemu.
    \item \textbf{Rejestracja} - proces, w którym administrator systemu tworzy nowe konto użytkownika.
    \item \textbf{Wylogowanie} - proces, w którym użytkownik przerywa korzystanie z aplikacji webowej.
    \item \textbf{Autoryzacja} - proces umożliwiający jednoznaczne zidentyfikowanie użytkownika i nadanie mu odpowiednich uprawnień.
    \item \textbf{Urządzenie mobilne} - smartfon, tablet lub inny przenośny sprzęt elektroniczny, który umożliwia korzystanie z aplikacji mobilnej. Jeżeli urządzenie jest wyposażone w system NFC, może również służyć do autoryzacji pracownika przy czytniku kart.
    \item \textbf{Autoryzacja mobilna} - proces, w którym użytkownik autoryzuje się za pomocą aplikacji mobilnej na swoim urządzeniu lub karty zbliżeniowej.
    \item \textbf{Panel} - interfejs graficzny, który umożliwia użytkownikowi korzystanie z funkcjonalności systemu.
    \item \textbf{Pracownik} - osoba wykonująca pracę lub zatrudniona na stałe w organizacji.
    \item \textbf{Użytkownik} - osoba zarejestrowana w systemie, która ma dostęp do jakichkolwiek jego funkcjonalności.
    \item \textbf{SuperUser} - użytkownik należący do grupy administratorów, który posiada pełne uprawnienia do zarządzania użytkownikami i konfiguracji systemu.
    \item \textbf{Administrator} - użytkownik z uprawnieniami do zarządzania managerami, użytkownikami, harmonogrami i strukturą organizacyjną.
    \item \textbf{Manager} - użytkownik, który zarządza zespołem pracowników. Może przypisywać zadania, zmieniać harmonogramy i monitorować pracę podległych mu pracowników.
    \item \textbf{Harmonogram, grafik pracy} - plan, który określa godziny pracy pracowników w danym okresie czasu.
    \item \textbf{Harmonogram indywidualny} - harmonogram pracy dla konkretnego pracownika.
    \item \textbf{Wniosek urlopowy} - prośba pracownika o przyznanie urlopu w określonym terminie.
    \item \textbf{Timesheet} - zestawienie działań podjętych przez pracownika w danym okresie czasu. Wypełniany przez pracownika pod koniec każdego dnia pracy. Akceptowany przez managera.
    \item \textbf{Raport czasu pracy} - zestawienie godzin pracy i czasu spędzonego na poszczególnych zadaniach przez pracownika w danym okresie czasu.
    \item \textbf{Archiwizacja danych pracownika} - proces przenoszenia danych pracownika do archiwum po zakończeniu umowy pracy. Wymaga usunięcia dostępu do systemu oraz adresu e-mail pracownika.
    \item \textbf{Stanowisko} - określona rola pracownika w organizacji, np. programista, tester, manager.
    \item \textbf{Jednostka organizacyjna} - zbiór działów podejmujących wspólne cele w organizacji.
    \item \textbf{Dział} - jednostka organizacyjna, która zajmuje się określonym obszarem działalności jednostki organizacyjnej.
    \item \textbf{Ustawienia systemu} - konfiguracja systemu, która umożliwia dostosowanie go do potrzeb organizacji.
\end{itemize}

\end{document}