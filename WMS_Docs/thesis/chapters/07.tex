\chapter{Podsumowanie i wnioski}

\section{Wnioski}

Celem pracy było zaprojektowanie i implementacja systemu do zarządzania personelem w firmie. Umożliwia on zarządzanie pracownikami, ich zadaniami, czasem pracy i strukturą firmy w sposób zautomatyzowany. Wymierne korzyści z zastosowania systemu to zwiększenie efektywności pracy, zwiększenie kontroli nad zadaniami i pracownikami oraz zwiększenie przejrzystości ról w firmie. Przeprowadzona analiza wymagań pozwoliła na stworzenie systemu, który spełnia oczekiwania użytkowników, co potwierdziły przeprowadzone testy.

Wynikiem pracy jest system, który spełnia założenia projektowe i udostępnia użytkownikom sprecyzowane w wymaganiach funkcjonalności. Jest on bezpieczny, łatwy w obsłudze i automatyzuje szereg procesów, które wcześniej były wykonywane ręcznie. Aplikacja mobilna pozwala na dostęp do systemu z każdego miejsca i w każdym czasie, a dzięki funkcji dodawania kart zbliżeniowych niweluje konieczność ręcznego wpisywania danych lub użycia czytników podłączanych do komputera. Użycie mikrokontrolerów pozwala na automatyzację zapisu czasu pracy pracowników, co znacznie ułatwia jego śledzenie. Dzięki ich zastosowaniu możliwa jest również obsługa kontroli dostępu do pomieszczeń.

System został zrealizowany w wersji podstawowej, jednak posiada wiele możliwości rozwoju. Udało się zrealizować funkcjonalności dotyczące: rejestracji czasu pracy, zarządzania pracownikami, harmonogramami pracy, zadaniami, czasem pracy i strukturą firmy. Nie udało się zrealizować funkcjonalności dotyczących raportów, wniosków oraz automatycznego śledzenia czasu pracy pracowników. Wszystkie one mogą jednak zostać dodane w przyszłości, co znacznie zwiększyłoby użyteczność systemu. Implementacja kontroli dostępu do pomieszczeń została zrealizowana w ograniczonym zakresie, jednak możliwe jest jej rozbudowanie.

\subsection{Problemy napotkane podczas pracy}

Podczas pracy napotkano kilka problemów, które znacząco ją utrudniły i wydłużyły. Największym problemem było to, że przy projektowaniu systemu nie wzięto pod uwagę stopnia jego złożoności. Poskutkowało to koniecznością zrezygnowania z części funkcjonalności lub ograniczeniem ich zakresu na rzecz dotrzymania terminu.

W założeniach aplikacji mobilnej, miała ona umożliwiać autoryzację użytkownika poprzez przyłożenie urządzenia do czytnika kart zbliżeniowych. Niestety, przez decyzje podjęte na etapie projektowania komunikacji między układem mikrokontrolera a serwerem, nie udało się zrealizować tej funkcjonalności. Autoryzacja opiera się o numery seryjne kart, co znacznie ułatwia proces ich odczytania i rejestracji. Urządzenie mobilne musiało by być w stanie emulować te numery, jednakże nie jest to możliwe bez modyfikacji systemu operacyjnego.

Framework \texttt{Expo} użyty do stworzenia aplikacji mobilnej udostępniał funkcję odczytywania zmiennych środowiskowych. Niestety, po pewnym czasie funkcja przestała odświeżać wartości, co wiązało się z koniecznością częstej, wielokrotnej kompilacji aplikacji. Problem ten nie został rozwiązany, co znacznie wydłużyło pracę nad aplikacją mobilną.

\subsection{Ocena dobranych technologii po zakończeniu pracy}

Technologie wybrane do realizacji systemu okazały się być idealne, ponieważ w sposób znaczący przyspieszyły pracę nad projektem. Mikrokontroler, na którym zainstalowano system \texttt{MicroPython} okazał się być bardo prosty w obsłudze i umożliwił szybkie zaimplementowanie funkcjonalności związanych z odczytem kart zbliżeniowych. \texttt{Spring Framework} nie sprawił, że system stał się zbyt skomplikowany, a wręcz przeciwnie - pozwolił na jasne zdefiniowanie struktury aplikacji i łatwe zarządzanie nią. Framework \texttt{Vue.js} zrobił na autorze bardzo dobre wrażenie swoją prostotą i intuicyjnością. W porównaniu z innymi frameworkami, takimi jak \texttt{Next.js}, okazał się być znacznie przyjaźniejszy, jawnie oddzielając warstwę prezentacji, logiki i danych. \texttt{Expo} okazało się najbardziej uciążliwą technologią, jednakże poza wymienionym nie sprawiło innych problemów. Użycie innej technologii, np. \texttt{Flutter} mogłoby przyspieszyć pracę nad aplikacją mobilną, jednakże konieczność nauki nowego frameworka mogłaby wydłużyć czas potrzebny na jej stworzenie.

\section{Perspektywy rozwoju}

System posiada wiele możliwości rozwoju, które znacznie zwiększyłyby jego użyteczność. Pierwszym krokiem, jaki należy podjąć jest zwiększenie ilości czytników kart zbliżeniowych - obecnie w systemie działa tylko jeden, ale możliwe jest łatwe podłączenie większej ich ilości. Następnie należałoby zająć się brakującymi częściami: raportami, wnioskami i automatycznym śledzeniem czasu pracy pracowników. Raporty powinny być generowane w formie plików PDF (ang. \english{Portable Document Format}), które można by było wydrukować lub przesłać mailem. Do wniosków - oprócz możliwości ich składania i akceptacji - dobrze byłoby dodać możliwość wymieniania się wiadomościami w formie czatu. Automatyczne śledzenie czasu pracy może być zrealizowane algorytmicznie, wywołując odpowiednie funkcje w określonych momentach. Możliwe, że dobrym wyborem było by zastosowanie algorytmów uczenia maszynowego, które nauczyły by się rozpoznawać co pracownik robi w danym momencie. Kontrola dostępu do pomieszczeń również wymaga rozbudowy - aktualnie użytkownicy mogą autoryzować się przy każdym czytniku bez ustalania dostępu do pomieszczeń. W przyszłości warto byłoby dodać możliwość ustalania uprawnień dla poszczególnych użytkowników. W dalszej perspektywie można rozszerzać system o kolejne funkcjonalności znane z systemów HCM.

\section{Podsumowanie}

Samodzielne stworzenie systemu, który spełnia podane wymagania, jest bardzo czasochłonne. Wymaga od programisty bardzo dużej wiedzy, umiejętności i zaangażowania. Duże zespoły są w stanie tworzyć systemy o wiele bardziej rozbudowane i złożone w krótszym czasie - co widać po rozwiązaniach dostępnych na rynku. Każdy z członków zespołu wnosi do projektu swoje doświadczenie, wiedzę i specjalizację, dzięki czemu praca nad projektem jest bardziej efektywna i przyjemna.

Wykonanie projektu pozwoliło na zrozumienie procesów zachodzących w firmach i ukazało, jakie korzyści przynosi zastosowanie systemu informatycznego. Praca nad systemem pozwoliła na zdobycie ogromnych zasobów wiedzy - tak technicznej, jak i biznesowej. Projekt okazał się być dużym wyzwaniem, ale również przyniósł wiele satysfakcji. Udało się stworzyć system, który spełnia założenia projektowe i jest gotowy do dalszego rozwoju. Praca nad projektem pozwoliła na zdobycie cennego doświadczenia i wiedzy, które przyniosą wymierne korzyści w przyszłości. System posiada wiele możliwości rozwoju, które mogą znacznie zwiększyć jego użyteczność i funkcjonalność. Warto kontynuować prace nad nim, aby w pełni wykorzystać jego potencjał.