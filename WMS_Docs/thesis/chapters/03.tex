\chapter{Wymagania i narzędzia}
\label{ch:wymagania-i-narzedzia}

% \begin{itemize}
% \item wymagania funkcjonalne i niefunkcjonalne
% \item przypadki użycia (diagramy UML) -- dla prac, w których mają zastosowanie
% \item opis narzędzi, metod eksperymentalnych, metod modelowania itp.
% \item metodyka pracy nad projektowaniem i implementacją -- dla prac, w których ma to zastosowanie
% \end{itemize}

\section{Założenia projektowe}

% \note{W tej części pracy należy opisać, jakie założenia przyjęto podczas tworzenia systemu. Należy zaznaczyć, jakie funkcje systemu są najważniejsze, a które mogą być pominięte.}

Podczas tworzenia systemu przyjęto zbiór założeń, które miały na celu określenie zakresu projektu oraz jego funkcjonalności.

\subsection{Wymagania funkcjonalne}

Projekt systemu zakłada spełnienie następujących wymagań funkcjonalnych:

\begin{itemize}
    \item zarządzanie pracownikami: dodawanie, usuwanie, edycja, przypisywanie ról,
    \item zarządzanie zadaniami: dodawanie, usuwanie, edycja, konieczność akceptacji przez przełożonego,
    \item zarządzanie czasem pracy: harmonogramowanie czasu pracy,
    \item zarządzanie strukturą firmy: dodawanie, usuwanie, edycja działów, stanowisk i ról,
    \item rejestracja czasu pracy pracowników: odczyt kart zbliżeniowych, zapisywanie czasu pracy,
    \item kontrola dostępu do pomieszczeń: autoryzacja kart zbliżeniowych, przyznawanie dostępu,
    \item generowanie raportów: raporty z czasu pracy, zadań, działania systemu,
    \item wnioski: składanie, akceptacja, odrzucanie, komunikacja,
    \item automatyczne śledzenie czasu pracy pracowników: rozpoznawanie aktywności pracownika.
\end{itemize}

\subsection{Wymagania niefunkcjonalne}

Projekt systemu zakłada spełnienie następujących wymagań niefunkcjonalnych:

\begin{itemize}
    \item system dostępny jest przez 24 godziny na dobę, 7 dni w tygodniu, z wyjątkiem przerw na konserwację,
    \item aplikacja działa płynnie i bez zacięć, nawet przy dużej ilości użytkowników,
    \item system jest bezpieczny i odporny na ataki z zewnątrz,
    \item aplikacja mobilna działa na systemach Android i iOS,
    \item aplikacja jest intuicyjna i łatwa w obsłudze,
    \item system jest łatwy w utrzymaniu i rozbudowie,
    \item system jest zgodny z obowiązującymi przepisami prawa.
\end{itemize}

\section{Projekt systemu}

\subsection{Technologie}

Poniżej przedstawiono główne technologie wraz z uzasadnieniem ich wyboru oraz spis pozostałych technologii.

\subsubsection*{MySQL}

MySQL to relacyjna baza danych rozwijana aktualnie przez firmę Oracle.
Jej najważniejszymi cechami są: popularność, szybkość, niezawodność oraz łatwość w obsłudze.
Najnowsze wersje wspierają transakcje, widoki, procedury składowane i wiele innych funkcji, które zbliżają ją do baz danych typu Enterprise.
Dodatkowo oferuje lepsze prędkości odczytu danych, niż konkurencyjne rozwiązania takie jak PostgreSQL. \cite{bib:mysql}

Baza danych MySQL została wybrana ze względu na możliwości jakie oferuje, a także jej przejrzystość, co może ułatwić migrację na inne rozwiązanie w przyszłości.

\subsubsection*{Spring Framework}

Spring to platforma, która dostarcza rozwiązania do tworzenia aplikacji typu Enterprise w języku Java.
Jego asynchroniczna, nieblokująca architektura umożliwia tworzenie rozwiązań, obsługujących duże ilości danych, przy jednoczesnym zachowaniu wysokiej wydajności.
Składa się z wielu modułów obsługujących kluczowe aspekty działania systemu, takie jak: transakcje, bezpieczeństwo, obsługa danych, integracja z bazami danych, obsługa REST API, itp. \cite{bib:spring}

Spring Framework został wybrany ze względu na swoją popularność, wsparcie społeczności oraz bogatą dokumentację, co ułatwiło pracę nad projektem.

\subsubsection*{Vue.js}

Vue.js jest progresywnym frameworkiem JavaScript przeznaczonym do budowania interfejsów użytkownika. Charakteryzuje się lekkością, prostotą oraz wydajnością. Jego architektura oparta na komponentach umożliwia tworzenie skomplikowanych interfejsów, które są łatwe w utrzymaniu i rozbudowie. Dodatkowo oferuje udostępnienie aplikacji w formie SPA (ang. \english{Single Page Application}), dzięki któremu użytkownik widzi ją jako pojedynczą stronę internetową. Reakcje na interakcje użytkownika są szybkie i płynne, co zwiększa komfort korzystania z aplikacji przypominając niejako aplikacje desktopowe. \cite{bib:vuejs}

Vue.js został wybrany ze względu na chęć poznania tej technologii, względnie niski próg wejścia oraz dużą popularność wśród programistów i firm.

\subsubsection*{Expo}

Expo to platforma, która umożliwia tworzenie wieloplatformowych aplikacji mobilnych w językach JavaScript i TypeScript. Została stworzona na bazie {\bfseries React Native} i oferuje wiele gotowych rozwiązań dotyczących zarówno interfejsu użytkownika, jak i ustawień aplikacji. Platforma umożliwia również szybkie wdrożenie aplikacji dzięki wbudowanemu serwerowi deweloperskiemu oraz narzędziom do kompilacji i publikacji aplikacji na platformach Android i iOS używając EAS (ang. \english{Expo Application Service}). \cite{bib:expo}

Expo zostało wybrane ze względu na język programowania, który jest znany autorowi pracy, a także na możliwość szybkiego utworzenia i wdrożenia aplikacji na platformy mobilne.

\subsubsection*{Raspberry Pi Pico W}

Raspberry Pi Pico W jest najmniejszym modułem z rodziny Raspberry Pi, który może łączyć się z siecią. Posiada wbudowany kontroler Raspberry RP4020 oparty na architekturze ARM, co czyni go idealnym wyborem do zastosowań IoT. Układ wyposażony jest w 264 KB pamięci RAM, 2 MB pamięci flash oraz 26 pinów GPIO, dzięki którym bez problemu można podłączyć do niego wiele różnych czujników i urządzeń. Pico W posiada wbudowany moduł Wi-Fi oraz Bluetooth, co pozwala na łatwe łączenie się z siecią oraz innymi urządzeniami.

Raspberry Pi Pico W został wybrany ze względu na możliwość połączenia z siecią oraz duże możliwości rozbudowy. \cite{bib:picoW}

\subsubsection*{MicroPython}

MicroPython jest implementacją języka Python przeznaczona dla mikrokontrolerów i systemów wbudowanych, zoptymalizowaną pod kątem niskiego zużycia zasobów - co czyni ją idealnym wyborem dla urządzeń o ograniczonej mocy obliczeniowej i pamięci. MicroPython oferuje większość funkcji standardowego Pythona, co pozwala na szybkie i efektywne tworzenie oprogramowania dla mikrokontrolerów. Dzięki temu istnieje możliwość korzystania z dobrze znanych narzędzi i bibliotek, co znacząco przyspiesza proces tworzenia i testowania programu. \cite{bib:micropython}

MicroPython został wybrany ze względu na jego prostotę, elastyczność oraz możliwość szybkiego wdrożenia mikrokontrolera do systemu.

\subsubsection*{Inne technologie}

W projekcie wykorzystano również mniej znaczące technologie, które przedstawiono w tabelach \ref{tab:backend-tech}, \ref{tab:frontend-tech} oraz \ref{tab:embed-tech}.

\begin{tabularx}{\textwidth}{|p{4cm}|X|l|p{3cm}|}
    \caption{Biblioteki i frameworki wykorzystane w części backend}\label{tab:backend-tech}                                                                 \\
    \hline
    \textbf{Technologia} & \textbf{Opis}                                                                              & \textbf{Autor} & \textbf{Licencja}  \\
    \hline
    Spring Boot          & Framework do tworzenia aplikacji w języku Java \cite{bib:springBoot}                       & Spring         & Apache License 2.0 \\
    \hline
    Spring Security      & Framework do zarządzania bezpieczeństwem aplikacji \cite{bib:springSecurity}               & Spring         & Apache License 2.0 \\
    \hline
    Springdoc OpenAPI    & Biblioteka do generowania dokumentacji API oraz dostępu do Swagger-ui \cite{bib:springdoc} & Springdoc      & Apache License 2.0 \\
    \hline
    MySQL Connector      & Sterownik do łączenia się z bazą danych MySQL \cite{bib:mysqlConnector}                    & Oracle         & GPL 2.0            \\
    \hline
    Lombok               & Biblioteka do generowania kodu Java \cite{bib:lombok}                                      & Project Lombok & MIT                \\
    \hline
    JJWT                 & Biblioteka do obsługi tokenów JWT \cite{bib:jjwt}                                          & jwtk           & Apache License 2.0 \\
    \hline
\end{tabularx}

\begin{tabularx}{\textwidth}{|p{3cm}|X|l|l|}
    \caption{Biblioteki i frameworki wykorzystane w części frontend oraz aplikacji mobilnej}\label{tab:frontend-tech}                                                                \\
    \hline
    \textbf{Technologia}          & \textbf{Opis}                                                                                             & \textbf{Autor}   & \textbf{Licencja} \\
    \hline
    Axios                         & Biblioteka do wykonywania zapytań HTTP \cite{bib:axios}                                                   & Axios            & MIT               \\
    \hline
    Heroicons                     & Zestaw ikon SVG \cite{bib:heroicons}                                                                      & Tailwind Labs    & MIT               \\
    \hline
    PrimeVue                      & Biblioteka komponentów do Vue.js \cite{bib:primevue}                                                      & PrimeTek         & MIT               \\
    \hline
    Luxon                         & Biblioteka do obsługi dat i czasu \cite{bib:luxon}                                                        & Moment.js        & MIT               \\
    \hline
    Valibot                       & Biblioteka do walidacji formularzy \cite{bib:valibot}                                                     & Fabian Hiller    & MIT               \\
    \hline
    Vue i18n                      & Biblioteka do obsługi wielojęzyczności \cite{bib:vuei18n}                                                 & Intlify          & MIT               \\
    \hline
    Vue Router                    & Biblioteka do zarządzania trasami w aplikacji Vue.js \cite{bib:vueRouter}                                 & Vue              & MIT               \\
    \hline
    TailwindCSS                   & Biblioteka do stylowania aplikacji internetowych \cite{bib:tailwindcss}                                   & Tailwind Labs    & MIT               \\
    \hline
    Nativewind                    & Biblioteka umożliwiająca używanie TailwindCSS w aplikacjach opartych o React Native \cite{bib:nativewind} & Nativewind       & MIT               \\
    \hline
    React Native  Gesture Handler & Biblioteka do obsługi gestów w aplikacjach opartych o React Native \cite{bib:reactNativeGestureHandler}   & Software Mansion & MIT               \\
    \hline
    react native nfc manager      & Biblioteka do obsługi NFC w aplikacjach opartych o React Native \cite{bib:reactNativeNfcManager}          & RevtelTech       & MIT               \\
    \hline
\end{tabularx}

\begin{tabularx}{\textwidth}{|l|X|p{3cm}|p{3cm}|}
    \caption{Biblioteki wykorzystane w programie układu mikroprocesorowego}\label{tab:embed-tech}                                                                \\
    \hline
    \textbf{Technologia} & \textbf{Opis}                                                                       & \textbf{Autor}             & \textbf{Licencja}  \\
    \hline
    mfrc522              & Biblioteka do obsługi modułu RFID \cite{bib:mfrc522}                                & Daniel Perron              & MIT                \\
    \hline
    picozero             & Biblioteka do obsługi modułów peryferyjnych z Raspberry Pi Pico \cite{bib:picozero} & Raspberry Pi Foundation    & MIT                \\
    \hline
    Requests             & Biblioteka do wykonywania zapytań HTTP w Python \cite{bib:requests}                 & Python Software Foundation & Apache License 2.0 \\
    \hline
\end{tabularx}

\subsection{Narzędzia}

Podczas tworzenia systemu wykorzystano narzędzia, które przedstawiono w tabeli \ref{tab:tools}.

\begin{tabularx}{\textwidth}{|l|X|l|}
    \caption{Narzędzia wykorzystane podczas tworzenia systemu}\label{tab:tools}                                                                                                   \\
    \hline
    \textbf{Narzędzie} & \textbf{Opis}                                                                                                                       & \textbf{Producent} \\
    \hline
    Docker             & Narzędzie do uruchamiania kontenerów, w projekcie wykorzystane do uruchomienia serwera bazy danych \cite{bib:docker}                & Docker Inc.        \\
    \hline
    Git                & System kontroli wersji \cite{bib:git}                                                                                               & Linus Torvalds     \\
    \hline
    Visual Studio Code & Edytor kodu wykorzystany przy tworzeniu części frontend oraz aplikacji mobilnej \cite{bib:vscode}                                   & Microsoft          \\
    \hline
    IntelliJ IDEA      & Zintegrowane środowisko programistyczne do języków Java oraz Kotlin, wykorzystane przy tworzeniu części backend \cite{bib:intellij} & JetBrains          \\
    \hline
    Figma              & Narzędzie do projektowania interfejsów użytkownika \cite{bib:figma}                                                                 & Figma              \\
    \hline
    Zeplin             & Narzędzie do współpracy nad projektami interfejsów użytkownika, umożliwiające generowanie stylów CSS \cite{bib:zeplin}              & Zeplin             \\
    \hline
    Thonny             & Zintegrowane środowisko programistyczne dla języka Python na mikrokontrolery \cite{bib:thonny}                                      & Thonny             \\
    \hline
    PlantUML           & Narzędzie do tworzenia diagramów UML \cite{bib:plantuml}                                                                            & PlantUML           \\
    \hline
\end{tabularx}

% \section{Przypadki użycia}

% \note{W tej części pracy należy przedstawić diagramy UML przypadków użycia systemu oraz je opisać.}
