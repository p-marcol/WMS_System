%%%%%%%%%%%%%%%%%%%%%%%%%%%%%%%%%%%%%%%%%
% Academic Title Page
% LaTeX Template
% Version 2.0 (17/7/17)
%
% This template was downloaded from:
% http://www.LaTeXTemplates.com
%
% Original author:
% WikiBooks (LaTeX - Title Creation) with modifications by:
% Vel (vel@latextemplates.com)
%
% License:
% CC BY-NC-SA 3.0 (http://creativecommons.org/licenses/by-nc-sa/3.0/)
% 
% Instructions for using this template:
% This title page is capable of being compiled as is. This is not useful for 
% including it in another document. To do this, you have two options: 
%
% 1) Copy/paste everything between \begin{document} and \end{document} 
% starting at \begin{titlepage} and paste this into another LaTeX file where you 
% want your title page.
% OR
% 2) Remove everything outside the \begin{titlepage} and \end{titlepage}, rename
% this file and move it to the same directory as the LaTeX file you wish to add it to. 
% Then add \input{./<new filename>.tex} to your LaTeX file where you want your
% title page.
%
%%%%%%%%%%%%%%%%%%%%%%%%%%%%%%%%%%%%%%%%%

%----------------------------------------------------------------------------------------
%	PACKAGES AND OTHER DOCUMENT CONFIGURATIONS
%----------------------------------------------------------------------------------------
\documentclass[11pt,a4paper,titlepage,hidelinks]{article}

\usepackage[utf8]{inputenc}
\usepackage[polish]{babel}
\usepackage[T1]{fontenc}
\usepackage{float}
\usepackage{hyperref}
\usepackage{amsmath}
\usepackage{hyperref}
\usepackage[margin=1in]{geometry}
\usepackage{enumitem}

\usepackage{mathpazo} % Use the Palatino font by default


\newcommand\myemptypage{
	\null
	\thispagestyle{empty}
	\newpage
}

\newcommand\pageCounterMinusOne{
	\addtocounter{page}{-1}
}

\title{\Huge{Projekt inżynierski}\\\LARGE{Workforce Management System}}
\author{Piotr Marcol}
\date{\today}


\begin{document}

\selectlanguage{polish}

\begin{titlepage} % Suppresses displaying the page number on the title page and the subsequent page counts as page 1

    \newcommand{\HRule}{\rule{\linewidth}{0.5mm}} % Defines a new command for horizontal lines, change thickness here

    \center % Centre everything on the page

    %------------------------------------------------
    %	Headings
    %------------------------------------------------

    \textsc{\LARGE Politechnika Śląska}\\[0.5cm] % Main heading such as the name of your university/college

    \textsc{\Large Wydział Automatyki, Elektroniki i Informatyki}\\[1.5cm] % Major heading such as course name

    \textsc{\Large Projekt inżynierski}\\[0.5cm] % Minor heading such as course title

    \textsc{\LARGE Analiza tematu}\\[0.2cm] % Minor heading such as course title

    %------------------------------------------------
    %	Title
    %------------------------------------------------

    \HRule\\[0.4cm]

    { \LARGE System zarządzania personelem \\[0.2cm]
    Workforce management system }\\[0.2cm] % Title of your document

    \HRule\\[1.5cm]

    %------------------------------------------------
    %	Author(s)
    %------------------------------------------------

    \begin{minipage}{0.5\textwidth}
        \begin{flushleft}
            % \large
            Piotr Marcol\\
            Inf. SSI, ISMIP\\
            Rok akademicki 2024/2025
        \end{flushleft}
    \end{minipage}
    % ~
    \begin{minipage}{0.4\textwidth}
        \begin{flushright}
            % \large
            \textit{Promotor}\\
            dr inż. Marcin Połomski
        \end{flushright}
    \end{minipage}



    %------------------------------------------------
    %	Logo
    %------------------------------------------------

    % \vfill\vfill
    % \includegraphics[width=0.5\textwidth]{polslLogo.jpeg}\\[1cm] % Include a department/university logo - this will require the graphicx package

    %----------------------------------------------------------------------------------------

    %------------------------------------------------
    %	Date
    %------------------------------------------------

    \vfill\vfill\vfill % Position the date 3/4 down the remaining page

    {\large\today} % Date, change the \today to a set date if you want to be precise

    \vfill % Push the date up 1/4 of the remaining page

\end{titlepage}

\newpage

\tableofcontents

\newpage

% \newpage

\section{Opis i wizja systemu}

Celem pracy jest zaprojektowanie i implementacja systemu informatycznego z aplikacją webową, który będzie służył do zarządzania personelem. System ma implementować część funkcjonalności ze zbioru procesów zarządzania znanych jako Workforce Management. Jego głównym zadaniem będzie ułatwienie monitorowania i zarządzania czasem pracy pracowników, a także ustalanie grafików pracy, zarządzanie urlopami i zarządzanie strukturą organizacyjną firmy. Identyfikacja pracowników będzie odbywać się poprzez system autoryzacji, a dostęp do poszczególnych funkcji systemu będzie zależał od przypisanych uprawnień. Czas pracy pracowników będzie rejestrowany poprzez układ czytników kart zbliżeniowych zainstalowanych w siedzibie firmy.

System ma być dostępny dla pracowników, menedżerów oraz administratorów. Każda z tych grup będzie miała dostęp do innych jego części. Pracownicy będą mogli sprawdzać swoje grafiki, zgłaszać wnioski urlopowe oraz sprawdzać swoje statystyki czasu pracy. Menedżerowie będą mieli dostęp do funkcji pracowników, a także do zarządzania grafikami pracy swoich podwładnych i raportów dotyczących czasu ich pracy. Administratorzy będą mieli dostęp do wszystkich funkcji systemu, w tym do zarządzania strukturą organizacyjną firmy oraz do zarządzania kontami użytkowników.

\section{Analiza dziedziny}

Kontrola czasu pracy pracowników jest jednym z kluczowych elementów zarządzania personelem w każdej firmie. Wpływa ona na efektywność pracy, a także na zadowolenie pracowników. W zależności od wielkości firmy oraz od jej specyfiki, metody kontroli czasu pracy mogą się różnić. W małych firmach, gdzie liczba pracowników jest niewielka, kontrola czasu pracy może być prowadzona w sposób tradycyjny, na przykład poprzez karty czasu pracy. Duże firmy z kolei mogą korzystać z bardziej zaawansowanych systemów informatycznych, które pozwalają na automatyzację procesów związanych z zarządzaniem personelem. W obu przypadkach celem jest zapewnienie, aby pracownicy byli obecni w miejscu pracy w określonym czasie, a także aby ich czas pracy był zgodny z obowiązującymi przepisami. Prowadzenie kontroli czasu pracy jest obowiązkowe dla pracodawców, a nieprzestrzeganie przepisów jest uznawane za naruszenie praw pracowniczych, co może skutkować karą finansową wysokości zgodnej z art. \nolinebreak 281 pkt \nolinebreak 6 Kodeksu Pracy.

\subsection{Tradycyjne metody kontroli czasu pracy}

Są to metody zarządzania personelem, które w ogóle lub w bardzo małym stopniu korzystają z technologii informatycznych. Większość z nich polega na ręcznym wprowadzaniu danych oraz na przechowywaniu ich w formie papierowej lub arkuszach kalkulacyjnych. W dzisiejszych czasach są one uważane za przestarzałe i nieefektywne, jednakże wciąż są stosowane w wielu firmach. Przykłady takich metod to:

\begin{itemize}
    \item \textbf{Indywidualne karty czasu pracy} - pracownik zaznacza swoją obecność na kartce papieru, a następnie podpisuje ją. Przy tej metodzie wymagana jest ręczna korekta błędów oraz przepisywanie danych do systemu. Zaletą tej metody jest jej prostota i niski koszt, jednakże jest ona mało efektywna i podatna na błędy. Dodatkowo wymaga przechowywania dużej ilości papierowych dokumentów oraz ich archiwizacji.
    \item \textbf{Arkusz kalkulacyjny} - metoda polegająca na prowadzeniu arkusza kalkulacyjnego, w którym podobnie jak w przypadku kart czasu pracy zapisywane są godziny pracy pracowników. Arkusz jest następnie przesyłany do działu kadr, który na jego podstawie oblicza wynagrodzenie pracownika. Metoda ta jest bardziej efektywna od kart czasu pracy, jednakże nadal wymaga ręcznego wprowadzania danych oraz jest podatna na błędy. Dodatkowo może pojawić się problem z ochroną danych osobowych innych pracowników. Rozwiązaniem może być zastosowanie osobnych arkuszy dla każdego pracownika, jednakże wtedy ilość danych diametralnie rośnie.
    \item \textbf{Harmonogram pracy} - metoda polegająca na ustalaniu grafików pracy pracowników na dany okres czasu. Grafik jest następnie wywieszany w widocznym miejscu, aby pracownicy mieli do niego dostęp. W przypadku zmian w grafiku, pracownik musi zostać o tym poinformowany przez przełożonego. Metoda ta jest stosowana w firmach, w których praca odbywa się w systemie zmianowym - na przykład w sklepach czy restauracjach. Wadą tej metody jest konieczność ręcznego wprowadzania zmian w grafiku oraz brak możliwości bezpośredniego dostępu do niego przez pracownika poza miejscem pracy. Nie pozwala ona również na monitorowanie czasu pracy pracowników w czasie rzeczywistym.
\end{itemize}

Połączenie tych metod daje możliwość stworzenia bardziej efektywnego systemu kontroli czasu pracy, jednakże wciąż nie jest to optymalne rozwiązanie. Dodatkowo korzystanie z papierowych kart i arkuszy jest obarczone ryzykiem utraty danych, nieautoryzowanego dostępu i brakiem możliwości szybkiego reagowania na zmiany w grafiku pracy.

\subsection{Systemy zarządzania kapitałem ludzkim (HCM)}

Są to systemy informatyczne, które pozwalają na zarządzanie personelem w sposób zautomatyzowany. Zawierają w sobie szereg funkcji związanych z zarządzaniem czasem pracy, rekrutacją, szkoleniami, wynagrodzeniem oraz rozwojem pracowników. Ich główną częścią jest moduł Workforce Management. Systemy te są zazwyczaj dostępne w formie aplikacji webowej, co pozwala na łatwy dostęp z dowolnego miejsca i urządzenia. Wadą jest ich koszt oraz konieczność przeszkolenia pracowników w zakresie ich obsługi. Przykłady takich systemów to:

\begin{itemize}
    \item \textbf{Oracle HCM Cloud} - kompletne rozwiązanie chmurowe firmy Oracle, które łączy w sobie funkcje zarządzania personelem, procesami kadrowymi, rekrutacyjnymi i płacowymi. Jest używany m. in. przez FUJIFILM, Deutsche Bahn, czy Fujitsu.
    \item \textbf{SAP SuccessFactors HCM} - rozwiązanie chmurowe firmy SAP, które oferuje szereg funkcji w zakresie HR. Zawiera w sobie moduły do zarządzania procesami kadrowymi, rekrutacyjnymi, szkoleniowymi, płacowymi i analitycznymi. Jest używany m. in. przez Microsoft, Nestle, Allianz.
    \item \textbf{MintHCM} - oprogramowanie firmy eVolpe oparte o otwartoźródłowe systemy CRM. Oferuje szereg funkcji związanych z zarządzaniem personelem, takich jak rekrutacja, szkolenia, oceny pracownicze, czy zarządzanie czasem pracy i urlopami.  Korzystają z niego m. in. Empik, Poczta Polska, czy Asseco.
\end{itemize}

Głównym powodem, dla którego firmy decydują się wdrożyć systemy HCM jest ich pozytywny wpływ na efektywność pracy, a co za tym idzie - zwiększenie przychodów. Dobrze zaprojektowany system, który pozwala na załatwienie wielu spraw formalnych oraz administracyjnych w jednym miejscu ułatwia pracownikom codzienną pracę, pozwala na szybsze reagowanie na zmiany w organizacji i daje jasny wgląd do danych dotyczących ich wydajności. Dla kadry, system umożliwia monitorowanie działań i wyników pracowników, co może przełożyć się na premie i awanse.

Często w rozwiązaniach HCM brakuje funkcji związanej z przyznawaniem dostępów oraz kontrolą wejść i wyjść pracowników z firmy. W takich przypadkach konieczne jest zintegrowanie systemu HCM z systemem kontroli dostępu, co zwiększa koszty i skomplikowanie systemu. Dodatkowo, systemy HCM są zazwyczaj dostępne tylko w formie chmurowej, co może być problemem dla firm, które chcą mieć pełną kontrolę nad swoimi danymi.

\subsection{Systemy kontroli dostępu}

Celem systemów kontroli dostępu jest zapewnienie bezpieczeństwa w firmie poprzez kontrolę wejść i wyjść pracowników oraz gości. Ich głównym zadaniem jest zapewnienie bezpieczeństwa pracownikom oraz chronienie mienia firmy. Pozwalają one na identyfikację osób przemieszczających się po budynku oraz na kontrolę dostępu do poszczególnych pomieszczeń. Zdarza się, że są one zintegrowane z alarmami oraz monitoringiem. Systemy te są zazwyczaj stosowane w dużych firmach, w których kontrola dostępu jest kluczowym elementem bezpieczeństwa. Takie systemy dostarczają m. in. firmy:

\begin{itemize}
    \item \textbf{Satel} - polska firma zajmująca się produkcją systemów alarmowych, monitoringowych i kontroli dostępu. Ich rozwiązania opierają się o technologię RFID. Możliwe jest ich wdrożednie lokalne oraz rozproszone.
    \item \textbf{Avigilon} - firma zajmująca się produkcją systemów monitoringu i kontroli dostępu. Ich rozwiązania opierają się o technologie bezprzewodowe oraz pinpady.
\end{itemize}

Systemy kontroli dostępu są zazwyczaj stosowane w firmach, w których bezpieczeństwo jest kluczowym elementem. Dla pracowników ich użytkowanie jest proste i intuicyjne, a dostęp do poszczególnych pomieszczeń jest szybki i wygodny. Niestety, systemy te nie oferują funkcji związanych z zarządzaniem personelem i czasem pracy. W takich przypadkach konieczne jest zintegrowanie systemu kontroli dostępu z systemem HCM, co zwiększa koszty i skomplikowanie systemu.

\subsection{Podsumowanie}

Analiza dziedziny pozwala na stwierdzenie, że istnieje zapotrzebowanie na system, który połączy w sobie funkcje zarządzania personelem, kontroli czasu pracy oraz kontroli dostępu. Obecne rozwiązania są albo przestarzałe i nieefektywne, albo nie zawierają w sobie wszystkich funkcjonalności. W związku z tym, zaprojektowanie i implementacja systemu, który będzie łączył w sobie wiele funkcji, może przynieść wymierne korzyści dla firm. Taki system pozwoli na zwiększenie efektywności pracy, bezpieczeństwa oraz ułatwi pracownikom codzienną pracę. Dodatkowo, pozwoli na zaoszczędzenie czasu i pieniędzy, które są obecnie wydawane na utrzymanie kilku różnych systemów.

\section{Wymagania funkcjonalne}

System składa się z kilku modułów, z których każdy odpowiada za inną część jego funkcjonalności. Wymagania funkcjonalne każdego z modułów są następujące:

\subsection{Moduł autoryzacji}

Jest to moduł dostępny dla wszystkich użytkowników, który pozwala na identyfikację użytkownika oraz na kontrolę dostępu do poszczególnych funkcji systemu.

\begin{itemize}
    \item {Istnienie konta SuperUser} - możliwość zalogowania się do systemu jako SuperUser należący do grupy administratorów i posiadający pełne uprawnienia.
    \item {Logowanie użytkownika} - możliwość zalogowania się do systemu za pomocą loginu i hasła, i automatyczne przekierowanie do odpowiedniego panelu w zależności od przypisanych uprawnień.
    \item {Zapomniane hasło} - możliwość wysłania linku do zmiany hasła na adres e-mail użytkownika.
    \item {Zmiana hasła} - możliwość zmiany hasła do systemu.
    \item {Wylogowanie użytkownika} - możliwość wylogowania się z systemu.
\end{itemize}

\subsection{Moduł autoryzacji mobilnej}

Moduł dostępny jest dla wszystkich użytkowników, którzy chcą korzystać z systemu za pomocą urządzenia mobilnego. Wymagane jest zainstalowanie aplikacji mobilnej oraz zarejestrowanie urządzenia w systemie.

\begin{itemize}
    \item {Rejestracja urządzenia} - możliwość zarejestrowania urządzenia mobilnego w momencie pierwszego logowania do aplikacji mobilnej.
    \item {Logowanie mobilne} - możliwość zalogowania się do systemu w aplikacji mobilnej poprzez podanie loginu i hasła lub zbliżenie karty dostępowej.
    \item {Autoryzacja zbliżeniowa} - możliwość autoryzacji użytkownika poprzez zbliżenie karty lub urządzenia mobilnego do czytnika.
\end{itemize}

\subsection{Moduł zarządzania czasem pracy}

Ten moduł dostępny odpowiada za zarządzanie czasem pracy pracowników. Dostępny jest dla wszystkich użytkowników.

\begin{itemize}
    \item {Przeglądanie grafiku pracy}.
    \item {Zgłaszanie wniosków urlopowych}.
    \item {Przeglądanie statystyk czasu pracy}.
    \item {Zgłaszanie nieobecności i zwolnień lekarskich}.
    \item {Zgłaszanie pracy poza firmą} - praca zdalna lub w delegacji.
    \item {Uzupełnienie danych o wykonanej pracy} - możliwość wypełniania timesheetu.
\end{itemize}

\noindent
Konta \textbf{menedżerów} i \textbf{administratorów} posiadają dodatkowe funkcje:

\begin{itemize}
    \item {Ustalenie grafiku pracy działu}.
    \item {Ustalenie indywidualnego grafiku pracy} - możliwość ustalenia grafiku pracy dla jednego pracownika.
    \item {Akceptacja pracy poza firmą} - możliwość akceptacji zgłoszeń pracy poza firmą.
    \item {Zarządzanie urlopami} - możliwość zarządzania podaniami o urlop pracowników.
    \item {Generowanie i przeglądanie raportów pracy}.
\end{itemize}

\subsection{Moduł zarządzania strukturą organizacyjną}

Moduł jest odpowiedzialny za zarządzanie strukturą organizacyjną firmy. Dostępny jest wyłącznie dla grupy \textbf{administratorów}.

\begin{itemize}
    \item {Dodawanie i rejestracja pracownika} do systemu.
    \item {Archiwizacja danych pracownika}.
    \item {Blokowanie i odblokowanie konta pracownika}.
    \item {Edycja danych pracownika}.
    \item {Dodawanie i rejestracja stanowiska}.
    \item {Edycja danych stanowiska}.
    \item {Dodawanie i rejestracja jednostki organizacyjnej}.
    \item {Edycja danych jednostki organizacyjnej}.
    \item {Dodawanie i rejestracja działu jednostki organizacyjnej}.
    \item {Edycja danych działu jednostki organizacyjnej}.
    \item {Przypisywanie pracowników do stanowisk}.
    \item {Przypisywanie działów do jednostek organizacyjnych}.
    \item {Przypisanie menedżerów do jednostek organizacyjnych}.
    \item {Przypisanie menedżerów do działów}.
    \item {Przypisanie użytkowników do działów}.
\end{itemize}

\subsection{Moduł zarządzania ustawieniami systemu}

Ten moduł dostępny jest jedynie dla \textbf{SuperUsera}. Powinien on umożliwiać zmianę ustawień wpływających na cały system, takich jak: język, strefa czasowa, format daty, itp.

\section{Wymagania niefunkcjonalne}

\subsection{Dostępność}

Aplikacja jest dostępna 24 godziny na dobę, 7 dni w tygodniu. W przypadku awarii, czas naprawy nie powinien przekraczać 2 godzin.

\subsection{Wydajność}

Aplikacja powinna być responsywna i działać płynnie, nawet przy dużej liczbie użytkowników. Czas wysłania odpowiedzi na żądanie nie powinien przekraczać 2 sekund.

\subsection{Wsparcie}

W dni robocze, w godzinach 8:00 - 15:00 dla użytkowników dostępne jest wsparcie techniczne w formie czatu online, rozmowy telefonicznej lub e-mail. Czas odpowiedzi na zgłoszenie nie powinien przekraczać godziny. Jeżeli problem nie zostanie rozwiązany w ciągu 2 dni roboczych, użytkownik ma prawo do odszkodowania. Producent zastrzega sobie prawo do zmiany maksymalnego czasu rozwiązania problemu.

System będzie poddawany regularnym testom wydajnościowym oraz bezpieczeństwa. Wszelkie błędy i usterki będą usuwane w miarę ich pojawiania się.

\subsection{Bezpieczeństwo}

Dane użytkowników przechowywane są w bazie danych. Wszelkie dane wrażliwe są szyfrowane, a możliwość ich odczytu mają jedynie uprawnieni pracownicy. Aplikacja jest zabezpieczona przed atakami typu SQL Injection, Cross-Site Scripting oraz Cross-Site Request Forgery. Wszelkie próby ataku są logowane i raportowane.

\subsection{Wdrożenie}

System będzie dostępny w formie aplikacji webowej oraz mobilnej. Wdrożenie systemu będzie odbywać się etapami, zgodnie z harmonogramem ustalonym przez producenta. Wszelkie zmiany w systemie będą wprowadzane w czasie ustalonym z klientem, aby nie zakłócać pracy użytkowników.

\end{document}