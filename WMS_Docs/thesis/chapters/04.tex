\chapter{Specyfikacja zewnętrzna}
\label{ch:04}

\section{Wygląd interfejsu użytkownika}

\note{Ta część powinna zawierać screeny z interfejsem użytkownika wraz z porównaniem do projektu graficznego.}

\section{Diagramy UML}

\note{Tutaj można dodać diagramy czynności i sekwencji.}

\section{Komponenty systemu}

\section{Użytkownicy i ich role}

\note{Ta część mogła by być opisana przy projektowaniu systemu.}

\section{Dostępność}

\subsection{Dostępność cyfrowa}

Aplikacja została zaprojektowana w taki sposób, aby możliwe było korzystanie z niej wyłącznie przy użyciu klawiatury. Użycie biblioteki komponentów \texttt{PrimeVue} pozwoliło na zapewnienie dostępności dla osób o ograniczonych możliwościach ruchowych. Komponenty umieszczone na stronie internetowej są zgodne z wytycznymi \texttt{WCAG 2.1}. \cite{bib:WCAG21}

\subsection{Dostępność językowa}

W celu zapewnienia dostępności systemu użytkownikom z różnych krajów i regionów, aplikacja dostarcza możliwość zmiany języka interfejsu użytkownika. W chwili obecnej dostępne są 2 języki: polski i angielski, jednakże dodanie kolejnego nie wymaga od programisty dużego nakładu pracy. Każdy z języków jest przechowywany w oddzielnym pliku \texttt{.json}, co pozwala na jego łatwą modyfikację lub dodanie nowego. Część pliku zawierającego tłumaczenia na język polski została przedstawiona na rysunku \ref{lst:pl}.

\begin{figure}[H]
    \begin{minted}[linenos,frame=lines]{json}
"form": {
    "save": "Zapisz",
    "cancel": "Anuluj",
    "fieldRequired": "To pole jest wymagane",
    "invalidFormat": "Nieprawidłowy format"
},
\end{minted}
    \caption{Fragment pliku z tłumaczeniami na język polski}
    \label{lst:pl}
\end{figure}