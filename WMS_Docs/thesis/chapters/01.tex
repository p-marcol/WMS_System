% \begin{itemize}
% \item wprowadzenie w problem/zagadnienie
% \item osadzenie problemu w dziedzinie
% \item cel pracy
% \item zakres pracy
% \item zwięzła charakterystyka rozdziałów
% \end{itemize}

\chapter{Wstęp}
\label{ch:wstep}

\section{Wprowadzenie do problematyki}

W każdym przedsiębiorstwie zatrudniającym pracowników musi zostać ustalona wewnętrzna hierarchia nazywana strukturą organizacyjną. Jest ona oficjalnym podziałem jednostek, komórek, stanowisk i pracowników w organizacji. W nowych przedsiębiorstwach najczęściej występuje nieformalny podział obowiązków, ale z czasem zaczyna klarować się jasny zakres działań poszczególnych osób. Następnie zaczyna się tworzyć struktura organizacyjna, która w pierwszym okresie działania firmy jest spłycona do dwóch poziomów: kierownictwa oraz działów. Taka hierarchia niestety nie sprzyja efektywnemu zarządzaniu, ponieważ obejmuje zbyt szerokie zakresy odpowiedzialności oraz powiązania między działami \cite{bib:zarzadzanie}. W takiej sytuacji konieczne jest wprowadzenie bardziej skomplikowanej struktury organizacyjnej, która umożliwi lepsze zarządzanie. Poniżej wymieniono najczęściej spotykane struktury organizacyjne \cite{bib:StrukturaOrganizacyjna}.

\begin{itemize}
    \item Struktura płaska - występuje w niej centralizacja władzy, w której wszyscy pracownicy podlegają jednemu kierownikowi. Najczęściej występuje w małych lub nowych firmach. Zapewnia szybki przepływ informacji i jest elastyczna.

    \item Struktura liniowa (prosta) - charakteryzuje się występowaniem trzech rodzajów stanowisk: kierowników, pracowników oraz dyrektorów. Każdy pracownik ma przydzielonego jednego przełożonego, który odpowiada przed swoim dyrektorem. W tej strukturze informacje są przekazywane tzw. drogą służbową. Aby pracownik mógł przekazać informacje do dyrekcji, muszą one przejść przez wszystkie szczeble zarządzania. Ten rodzaj struktury jest najbardziej popularny w przedsiębiorstwach franczyzowych.

    \item Struktura funkcjonalna - w tej strukturze każdy z pracowników odpowiada przed kilkoma przełożonymi. Pozwala ona zmniejszyć ilość obowiązków pojedynczego kierownika dzieląc je na kilka osób, które są specjalistami w danej dziedzinie. Każdy specjalista odpowiada przed dyrektorem lub menedżerem, odpowiedzialnym za całość działu.

    \item Struktura sztabowo-liniowa - jest połączeniem struktury liniowej oraz funkcjonalnej. Zespół składa się z jednego kierownika wspieranego przez kilku specjalistów oraz reszty pracowników. Nie istnieje pojedynczy sposób budowy tej struktury, ponieważ każdy z zespołów może mieć inną strukturę wewnętrzną.

    \item Struktura macierzowa (problemowa) - struktura rozdziela kierowników na dwa rodzaje: kierowników projektów oraz kierowników działów. Pracownicy podlegają jednocześnie kilku przełożonym, co zwiększa elastyczność firmy. W zespołach mogą wytworzyć się podgrupy pracowników pracujących nad jednym zadaniem.

    \item Struktura dywizjonalna - struktura określająca dokładną hierarchię w firmie. Najczęściej spotyka się ją w dużych firmach i korporacjach. Charakteryzuje się wyodrębnieniem działów, które są niemal samodzielne. Każdy z działów może mieć własną strukturę organizacyjną, odpowiednią do swoich potrzeb.
\end{itemize}

Struktura organizacyjna, jaką należy przyjąć w danym przedsiębiorstwie zależy ściśle od jego specyfiki oraz wielkości, jednakże w każdym przypadku powinna być ona jasno określona, aby umożliwić jak najskuteczniejsze zarządzanie firmą.

Kolejną krytyczną kwestią w zarządzaniu firmą jest kontrola czasu pracy pracowników. Wpływa ona na ich efektywność pracy i zadowolenie. W zależności od wielkości firmy oraz jej specyfiki, metody kontroli czasu pracy mogą się różnić. W firmach gdzie liczba pracowników jest niewielka, kontrola czasu pracy może być prowadzona w sposób tradycyjny; z kolei duże firmy mogą zastosować bardziej zaawansowane systemy informatyczne. Niezależnie od wybranego podejścia, jej celem jest zapewnienie, aby pracownicy byli obecni w miejscu pracy, w określonym czasie. Prowadzenie kontroli czasu pracy jest obowiązkowe dla pracodawców, a nieprzestrzeganie przepisów jest uznawane za naruszenie praw pracowniczych, co może skutkować karą finansową wysokości zgodnej z art. \nolinebreak 281 pkt \nolinebreak 6 Kodeksu Pracy \cite{bib:KodeksPracy}.

\section{Cel pracy}

% \note{W tej części pracy należy \textbf{jasno} opisać, jaki jest jej cel. Należy zaznaczyć, jakie cele zostaną osiągnięte po zakończeniu pracy.}

% Celem pracy jest analiza dziedziny, stwierdzenie zasadności stworzenia systemu, jego projekt oraz implementacja.

Celem pracy jest przede wszystkim projekt i implementacja systemu do zarządzania pracownikami w firmie. Docelowo system ma składać się z czterech głównych części:

\begin{itemize}
    \item serwera odpowiadającego za część biznesową,
    \item aplikacji webowej do zarządzania systemem,
    \item systemu mikroprocesorowego do autoryzacji i przyznania dostępu poprzez odczyt kart zbliżeniowych,
    \item aplikacji mobilnej umożliwiającej przypisywanie nowych kart dostępowych.
\end{itemize}

System ma umożliwiać zarządzanie pracownikami, ich zadaniami, czasem pracy oraz strukturą firmy, a także ma umożliwiać rejestrację czasu pracy pracowników oraz kontrolę dostępu do pomieszczeń. Wykonany projekt - implementując szereg funkcjonalności - ma być narzędziem wspierającym dział kadr automatyzując wiele procesów. Jednocześnie, jego architektura ma umożliwiać łatwą rozbudowę i integrację nowych funkcji w przyszłości.

W trakcie pracy zrealizowano również szereg zadań pobocznych takich jak analiza dziedziny, wyznaczenie wymagań funkcjonalnych i niefunkcjonalnych oraz wykonanie projektu graficznego.

\section{Zakres pracy}

W ramach pracy przeprowadzono analizę dziedziny, na podstawie której został stworzony projekt systemu. Następnie zostały wyznaczono wymagania funkcjonalne i niefunkcjonalne, które system musi spełniać oraz wykonany został jego projekt graficzny. Po zakończeniu fazy projektowania rozpoczęto prace nad implementacją. Zakończeniu prac towarzyszyło przeprowadzenie testów oraz ewentualne poprawki.

\section{Charakterystyka rozdziałów}

Praca składa się z wielu rozdziałów, z których każdy ma określone zadanie. Poniżej przedstawiono krótką charakterystykę każdego z nich.

W rozdziale drugim przedstawiono analizę dziedziny oraz przegląd dostępnych rozwiązań oraz. Wskazano w nim zasadność stworzenia systemu oraz rolę jaką ma on spełniać. Sformułowano i opisano również najważniejsze funkcjonalności, jakie powinien on posiadać.

Rozdział trzeci zawiera w sobie szczegóły dotyczące założeń, jakie musi spełniać system, aby był użyteczny. Opisano wymagania funkcjonalne i niefunkcjonalne, a także narzędzia i technologie, które zostały użyte do implementacji systemu.

Rozdział czwarty obejmuje wszelkie informacje dotyczące wyglądu i działania systemu, które są widoczne dla użytkownika. Zostały w nim opisane role użytkowników, dostępne funkcjonalności oraz sposób ich użytkowania. Znajduje się w nim również opis interfejsu graficznego oraz obsługi czytnika kart zbliżeniowych.

W piątym rozdziale zawarto informacje dotyczące implementacji systemu oraz jego architektury. Znajdują się w nim opisy najważniejszych klas, bazy danych, wybranych algorytmów i struktury połączeń układu czytnika kart zbliżeniowych.

W rozdziale szóstym skupiono się na opisie sposobów walidacji danych wejściowych oraz przeprowadzonych testach.

W ostatnim rozdziale przedstawiono wnioski wynikające z przeprowadzonych prac, problemy jakie w ich trakcie wystąpiły oraz propozycje możliwości rozwoju systemu w przyszłości.


