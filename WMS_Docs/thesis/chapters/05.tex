\chapter{Specyfikacja wewnętrzna}
\label{ch:05}

\section{Architektura systemu}

% \note{W tej części pracy należy opisać, jakie komponenty składają się na system. Należy zaznaczyć, jakie są relacje między nimi. Można również opisać, jakie wzorce projektowe zostały zastosowane. Głównie chodzi o to, aby czytelnik mógł zrozumieć, jak działa system.}

Kompletny system składa się z czterech głównych komponentów oraz bazy danych, które komunikują się ze sobą w celu zapewnienia pełnej funkcjonalności. Diagram architektury został przedstawiony na rysunku \ref{fig:architecture}. Kolejne podrozdziały opisują każdy z~komponentów oraz ich relacje.

% //www.plantuml.com/plantuml/png/RLBDSjCm4BxhANPooQayG0zC4-Wm37Ge3J-7JWvhUPjiB4iQQOEn6Ouy1C_18z07OvMaIGpu9Ez-FxkVT9K5asnRPnmz24Y0nAkrGCs9h73mPWeV5TCnUIPeq5l6BBrTWsFFRAQvOHynOLiA97qa24dMmPmQDhGnUFAoaDuH56NqAqT6nL-bRUoXUKBy_jMPYBR15S5a5XqtQ3S87LMXmmbyCw3V_QSjFTla9M26-yzKmImgqcAPTsc-c_yfjA5YvqyIrrzZs0lSb0uLZJ5tGQcZT2GiOcA_Wda8ikVDRb-3OhvGMkuGx8NI0rgbtbqlFe3s9mLuHrdqFD9-CVJYkO46anmTRjYcO8EN59nJLIYKARm9xTO_RS6FF-lXB2zi6V1WJPkil-QVHE3YulbnYhDrnus_xSTTdUCsTFpuOq3FT3BK-u81yYugQiq1NgZZW9J_BjfDyEYKVmbrg7T25UesUjG7PMeKh31hF7CSy50zfTClth_T9_zoNcxgN7zlg7Mu18MPA_9rUTH_0000
\begin{figure}[H]
    \centering
    \includegraphics[width=0.6\textwidth]{graf/systemScheme.png}
    \caption{Diagram architektury systemu}
    \label{fig:architecture}
\end{figure}

\subsection{Aplikacja webowa}

Głównym interfejsem użytkownika jest aplikacja webowa stworzona przy użyciu frameworka \texttt{Vue.js}. Zapewnia użytkownikowi dostęp do większości funkcji systemu, a~w~zależności od jego roli, umożliwia wykonanie innych czynności. Autoryzacja odbywa się poprzez tokeny JWT  (ang. \english{JSON Web Token}) \cite{bib:rfc7519}, które aplikacja przechowuje w pamięci przeglądarki. Wykorzystywana jest biblioteka \texttt{Vue Router} odpowiedzialna za ustalanie tras i~widoków, a także \texttt{Axios} do prostszego wykonywania zapytań do serwera. Frontend komunikuje się z serwerem aplikacyjnym wysyłając żądania HTTP (ang. \english{Hypertext Transfer Protocol}) i odbierając odpowiedzi w formacie \texttt{JSON} (ang. \english{JavaScript Object Notation}).

\subsection{Serwer}

Serwer aplikacyjny został stworzony przy użyciu frameworka \texttt{Spring Boot} korzystając z architektury Controller-Service-Repository \cite{bib:ControllerServiceRepository}. Zapewnia on komunikację między bazą danych, a pozostałymi komponentami systemu. Jest odpowiedzialny za przetwarzanie żądań klienckich oraz zwracanie odpowiedzi w formacie \texttt{JSON}. Serwer aplikacyjny odpowiada również za autoryzację i autentykację użytkowników oraz zarządzanie sesjami. Użytkownicy nie mają bezpośredniego dostępu do serwera.

\subsection{Aplikacja mobilna}

Aplikacja mobilna umożliwia użytkownikom przeglądanie swoich danych bez konieczności korzystania z przeglądarki internetowej. Jej głównym zadaniem jest jednak możliwość dodawania kart dostępowych dla poszczególnych użytkowników. Odbywa się to poprzez przyłożenie tagu NFC do telefonu z zainstalowaną aplikacją, a następnie przypisanie go do konkretnego użytkownika. Szczegółowy opis tego procesu został opisany w~rozdziale \ref{sec:addNfc}.

\subsection{Układ mikroprocesorowy}

Układ mikroprocesorowy jest odpowiedzialny za odczytywanie tagów NFC oraz przesyłanie informacji do serwera aplikacyjnego. Następnie serwer zwraca informację o przyznaniu dostępu. Układ mikroprocesorowy jest zasilany poprzez wbudowany w płytkę port Micro USB, dzięki czemu podłączenie go do zasilania jest bardzo proste. Dokumentacja techniczna mikrokontrolera \cite{bib:picoWdatasheet} precyzuje również podłączenie innego źródła zasilania bez użycia portu.

\section{Struktura bazy danych}

Baza danych obsługująca system składa się z kilkunastu tabel, które przechowują informacje o wszelkich danych w systemie. Na rysunku \ref{fig:dbDiagram} został przedstawiony jej uproszczony schemat, a w kolejnych podrozdziałach zostaną opisane najważniejsze tabele oraz relacje, jakimi są połączone.

\begin{figure}[H]
    \centering
    \includegraphics[width=\textwidth]{graf/dbDiagram.png}
    \caption{Uproszczony schemat bazy danych}
    \label{fig:dbDiagram}
\end{figure}

\subsection{Tabele użytkowników}

Główną tabelą przechowującą informacje o użytkownikach jest tabela \texttt{USERS}. Zawiera ona dane personalne, takie jak imię, nazwisko, adres e-mail i numer telefonu. Dodatkowo do tabeli wpisane są dane do logowania: nazwa użytkownika i hasło oraz informacja o~użytkowniku, który utworzył dany wpis. Każdy użytkownik ma przypisaną jedną z ról z~tabeli \texttt{DICT\_AUTHORITIES}, która określa jego uprawnienia w systemie. Relacją jeden do jednego jest połączona tabela \texttt{USER\_PAYOFF\_DATA} zawierające dane o koncie bankowym użytkownika. W tabeli \texttt{USERS} znajduje się również pole \texttt{is\_archived}, które określa, czy użytkownik jest aktywny w systemie. Przedstawienie graficzne tabel użytkowników znajduje się na rysunku \ref{fig:usersTable}.

\begin{figure}[H]
    \centering
    \includegraphics[width=0.7\textwidth]{graf/usersTable.png}
    \caption{Schemat tabel użytkowników}
    \label{fig:usersTable}
\end{figure}

\subsection{Tabele jednostek organizacyjnych}

Wszystkie jednostki organizacyjne są przechowywane w tabeli \texttt{UNITS} zawierającej dane o nazwie jednostki, jej opisie, jednostce nadrzędnej oraz dacie utworzenia. Aktywność jednostki określa pole \texttt{work\_ended}. Użytkownicy są przypisani do jednostki organizacyjnej poprzez tabelę \texttt{positions} będącą tabelą łącznikową między tabelami \texttt{USERS} i \texttt{UNITS}. Znajdują się w niej dane o dacie rozpoczęcia pracy na danym stanowisku oraz dacie zakończenia pracy, a także odwołanie do tabeli słownikowej, zawierającej nazwy stanowisk. Tabela \texttt{POSITIONS} jest szczególnie ważna przy odczytywaniu harmonogramów pracy. Schemat tabel jednostek organizacyjnych został przedstawiony na rysunku \ref{fig:organizationalUnitsTable}.

\begin{figure}[H]
    \centering
    \includegraphics[width=0.8\textwidth]{graf/unitsTable.png}
    \caption{Schemat tabel jednostek organizacyjnych}
    \label{fig:organizationalUnitsTable}
\end{figure}

\subsection{Tabele harmonogramów}
\label{ss:harmonogramy}


Na każdy z harmonogramów składa się pojedynczy rekord w tabeli \texttt{SCHEDULES} i~pewna liczba rekordów w tabeli \texttt{SCHEDULE\_BLOCKS}. Pierwsza z nich zawiera informacje o dacie rozpoczęcia i zakończenia harmonogramu, jego utworzenia oraz odwołanie do jednostki organizacyjnej lub użytkownika, którego dotyczy. Tabela \texttt{SCHEDULE\_BLOCKS} odpowiada za przechowywanie pojedynczych bloków czasowych w harmonogramie opisując jednostkę, której dotyczą, a także ich dzień i godzinę rozpoczęcia oraz zakończenia. Blok nie musi odpowiadać jednostce organizacyjnej, do której przypisany jest cały harmonogram. Tabele są powiazane relacją jeden do wielu - jeden harmonogram może zawierać wiele bloków czasowych. Przedstawienie tabel harmonogramów widoczne jest na rysunku \ref{fig:schedulesTable}.

\begin{figure}[H]
    \centering
    \includegraphics[width=0.7\textwidth]{graf/scheduleTable.png}
    \caption{Schemat tabel harmonogramów}
    \label{fig:schedulesTable}
\end{figure}

\subsection{Tabele kart dostępowych}

Karty dostępowe użytkowników są przechowywane w tabeli \texttt{ACCESS\_CARDS} zawierającej informacje o numerze seryjnym, właścicielu, opisie, typie karty oraz statusie. Dzięki ostatniej z właściwości możliwe jest przypisanie jednej karty dwóm użytkownikom w różnych okresach czasu. Każda autoryzacja użytkownika jest zapisywana w tabeli \texttt{USER\_ACCESS} wpisując do niej datę i godzinę, id użytkownika, id czytnika, id karty dostępowej oraz status autoryzacji. Umożliwia to późniejsze analizowanie historii dostępu. Wizualizacja tabel kart dostępowych znajduje się na rysunku \ref{fig:accessCardsTable}.

\begin{figure}[H]
    \centering
    \includegraphics[width=0.7\textwidth]{graf/acTable.png}
    \caption{Schemat tabel kart dostępowych}
    \label{fig:accessCardsTable}
\end{figure}

Tabela \texttt{DEVICES} przechowuje informacje o czytnikach kart - ich opis, datę ostatniego uruchomienia, adres MAC (ang. \english{Media Access Control address}) oraz symbol, którym identyfikuje się w systemie.

\subsection{Tabele słownikowe}

W bazie danych znajdują się trzy tabele słownikowe zawierające dane, które nie zmieniają się w czasie działania systemu. Nazwa każdej z nich poprzedzona jest prefiksem \texttt{DICT\_}. Są~to:
\begin{itemize}
    \item \texttt{DICT\_AUTHORITIES} zawierająca role użytkowników, równoznaczne z uprawnieniami,
    \item \texttt{DICT\_ACCESS\_CARD\_TYPES} zawierająca typy kart dostępowych dla łatwiejszego rozróżnienia tagów NFC,
    \item \texttt{DICT\_POSITION\_NAMES} zawierająca nazwy stanowisk przypisanych pracownikom.
\end{itemize}

Do tabel \texttt{DICT\_ACCESS\_CARD\_TYPES} i \texttt{DICT\_POSITION\_NAMES} mogą zostać dodane nowe rekordy, lecz nie jest możliwe ich usunięcie. Takie ograniczenie zapewnia integralność danych w systemie i zapobiega błędom w działaniu aplikacji.

Dane mogą dodawać jedynie administratorzy systemu.

\section{Modele i struktury danych}

% \note{W tej części pracy należy opisać, jakie modele danych zostały zastosowane w systemie. Należy zaznaczyć, jakie są relacje między nimi.}

Dzięki użyciu w projekcie JPA (ang. \english{Java Persistence API}) oraz Hibernate, struktury danych w systemie odpowiadają strukturom tabel w bazie danych - każda z tabel bazy danych jest mapowana na odpowiadającą jej klasę w systemie. Oprócz tego zostały zaimplementowane klasy pomocnicze oraz klasy DTO (ang. \english{Data Transfer Object}), które upraszczają logikę biznesową. Diagram klas został przedstawiony na rysunku \ref{fig:entityDiagram}.

\begin{figure}
    \centering
    \includegraphics[width=\textwidth]{graf/classDiagram.png}
    \caption{Diagram klas modeli}
    \label{fig:entityDiagram}
\end{figure}

\subsection{Klasy encji}

Zaimportowane zależności w projekcie umożliwiają tworzenie klas encji, które są mapowane na tabele w bazie danych. Każda z nich posiada adnotację \texttt{@Entity} - informującą JPA o tym, że klasa jest encją - oraz \texttt{@Table} z nazwą tabeli, z którą jest mapowana. Każde pole klasy, które odpowiada kolumnie w tabeli jest odpowiednio oznaczone. Do generowania metod \texttt{get} i \texttt{set} używane są adnotacje \texttt{@Getter} i \texttt{@Setter} z biblioteki \texttt{lombok}. W~klasach znajdują się również proste, publiczne metody przetwarzające dane, dzięki którym możliwe było uproszczenie logiki biznesowej.

\subsection{Klasy pomocnicze}

Aby zapewnić poprawne działanie systemu, niezbędne było zaimplementowanie klas pomocniczych. Nie są one mapowane na tabele w bazie danych, a ich celem jest uproszczenie logiki biznesowej oraz zwiększenie czytelności kodu. Dzielą się na dwie kategorie: narzędziowe oraz danych.

Klasy należące do pierwszej z kategorii zawierają jedynie metody statyczne i nie wymagają tworzenia instancji ich obiektu. Istnieją dwie takie klasy:
\begin{itemize}
    \item \texttt{DateUtils} --- zawierająca metody do zmiany dat z formatu \texttt{java.time.LocalDate} na \texttt{java.util.Date} oraz \texttt{java.sql.Date},
    \item \texttt{JwtTokenUtils} --- zawierająca metody do generowania, weryfikacji i wyłuskiwania danych z tokenów JWT.
\end{itemize}

Druga kategoria klas pomocniczych to klasy danych, które istnieją w celu uproszczenia przechowywania danych w systemie. Służą najczęściej jako kontenery na dane, które następnie zostaną przetworzone na obiekty klas DTO. Najważniejszą z nich jest klasa \texttt{Calendar}, reprezentująca harmonogramy w formie kalendarza. Przechowuje ona dane dla każdego dnia tygodnia w ustalonym przedziale czasowym i dodatkowo posiada metody do sortowania, nadpisywania dni oraz uzupełniania pustych miejsc w kalendarzu.

\subsection{Klasy DTO}

W pakiecie \texttt{apimodels}, odrębnym od głównego pakietu serwera znajdują się definicje klas DTO. Służą one do ustalania struktury obiektów przesyłanych między serwerem, a~klientem, zapewniając poprawne działanie systemu. Każda z nich jest zaimplementowana jako klasa Java, która posiadaja jedynie publiczne pola dostępowe. Nie jest na nich wykonywana żadna logika biznesowa, a dane są wpisywane bezpośrednio przed przesłaniem ich do odbiorcy.

Pakiet apimodels zawiera subpakiety odpowiadające kontrolerom, które wymagają przesłania pełnego obiektu. Są to:
\begin{itemize}
    \item \texttt{access\_card} --- modele kart dostępowych,
    \item \texttt{auth} --- modele autoryzacji i autentykacji,
    \item \texttt{position} --- modele stanowisk,
    \item \texttt{schedule} --- modele harmonogramów,
    \item \texttt{timesheet} --- modele kart pracy,
    \item \texttt{unit} --- modele jednostek organizacyjnych,
    \item \texttt{user} --- modele użytkowników.
\end{itemize}

\subsection{Klasy kontrolerów}

Klasy kontrolerów odpowiadają za obsługę żądań HTTP przesyłanych do serwera. Realizują one komunikację między klientem a serwerem wywołując odpowiednie metody i zwracając dane. Każda została oznaczona adnotacją \texttt{@RestController} oraz przypisano jej ścieżkę. Nie zawierają one logiki biznesowej, a jedynie wywołania odpowiednich metod serwisów. Przykład metody kontrolera został przedstawiony na listingu \ref{lst:controllerMethod}.

\begin{listing}[H]
    \begin{minted}[linenos, breaklines]{java}
@PostMapping("/add")
public ResponseEntity<?> addUnit(@RequestBody AddUnitRequest request) {
    try {
        unitService.addUnit(request);
        return ResponseEntity.ok("Unit added successfully");
    } catch (Exception e) {
        return ResponseEntity.badRequest().body("An error occurred");
    }
}
\end{minted}
    \caption{Metoda kontrolera jednostek organizacyjnych dodająca nową jednostkę}
    \label{lst:controllerMethod}
\end{listing}

\subsection{Interfejsy i klasy serwisów}

Serwisy są odpowiedzialne za logikę biznesową systemu. Każda z klas serwisów implementuje jego interfejs umożliwiając stworzenie kilku wariacji rozwiązania problemu. Dzięki temu możliwe jest łatwe dodanie nowych funkcji do systemu bez konieczności zmiany istniejącego już kodu. Metody interfejsów są publiczne, a niektóre klasy zawierają metody prywatne służące do przetwarzania danych. Adnotacją informującą platformę Spring o~tym, że klasa jest serwisem jest \texttt{@Service}.

\subsection{Repozytoria}

Częścią, która odpowiada za komunikację pomiędzy serwerem, a bazą danych są repozytoria. Są one interfejsami rozszerzającymi interfejs \texttt{JpaRepository} udostępniany przez Spring Data JPA. Metody w nich zawarte są nazywane zgodnie z konwencją określoną przez twórców, co umożliwia automatyczne generowanie zapytań SQL. Kod repozytorium harmonogramów ukazano na listingu \ref{lst:repositoryInterface}.

\begin{listing}[H]
    \begin{minted}[linenos, breaklines]{java}
public interface iScheduleRepository extends JpaRepository<Schedule, Long> {
    List<Schedule> findAllByUnitId(Long unitId);
    Schedule findFirstByUnitIdOrderByStartDateDesc(Long unitId);
    Schedule findFirstByUserIdOrderByStartDateDesc(Long userId);
    List<Schedule> findAllByUserId(Long id);
}
\end{minted}
    \caption{Przykład interfejsu repozytorium harmonogramów}
    \label{lst:repositoryInterface}
\end{listing}

\section{Wybrane algorytmy}

\subsection{Rejestracja i pierwsze logowanie użytkownika}

Rejestracja użytkownika może zostać dokonana jedynie przez administratora systemu. W tym celu powinien on wypełnić formularz rejestracyjny wprowadzając nazwę użytkownika oraz jego adres email. Po zatwierdzeniu formularza, w widoku wszystkich użytkowników pojawi się nowy rekord z danymi właśnie utworzonego użytkownika. Następnie administrator powinien przekazać nowo zarejestrowanemu użytkownikowi jego nazwę, którą musi wpisać na ekranie logowania - nie jest wymagane przy tym wpisywanie hasła. Po zatwierdzeniu formularza, i wysłaniu danych do serwera, zostanie sprawdzone czy użytkownik o podanej nazwie istnieje w bazie danych i czy ma przypisane hasło. Jeżeli nie, zwróci kod 206 - \texttt{Partial Content}, a aplikacja udostępni użytkownikowi możliwość ustawienia hasła. Po jego wpisaniu, potwierdzeniu i zatwierdzeniu formularza, użytkownik zostanie zalogowany do systemu. Diagram sekwencji pierwszego logowania użytkownika został przedstawiony na rysunku \ref{fig:login}, a szczegóły procesu zostały opisane w rozdziale \ref{ss:logowanie}.

\subsection{Logowanie}
\label{ss:logowanie}

Logowanie do systemu odbywa się poprzez przesłanie żądania HTTP z danymi logowania do serwera aplikacyjnego. Po jego otrzymaniu serwer zwraca się do bazy danych w celu znalezienia użytkownika o podanej nazwie. Jeżeli użytkownik nie istnieje, serwer zwraca kod błędu 401. W przeciwnym wypadku sprawdza, czy zahashowane hasło użytkownika jest zgodne z zapisanym w bazie. Jeżeli hasła się zgadzają, serwer generuje dwa tokeny JWT i zwraca je w odpowiedzi. Aplikacja kliencka zapisuje otrzymane tokeny w~pamięci lokalnej przeglądarki i przechodzi do widoku panelu głównego. Diagram czynności logowania został przedstawiony na rysunku \ref{fig:login}.

\begin{figure}[H]
    \centering
    \includegraphics[width=0.9\textwidth]{graf/loginSequence.png}
    \caption{Diagram sekwencji logowania}
    \label{fig:login}
\end{figure}
% //www.plantuml.com/plantuml/png/VP8nRiCm34LtdKB8dWju288CdOgYIz2PigX6iIC5jbJN7WFq43rBEzRtAeceK6pKcJpyh__9Hs_R04s8frf06NmZz-Dt7ohVELj9QAMBuaowBUqPN90FZNS1dMR9u4JQGLabHQ7G4411YtArWm6a1jUNXnMBMWaXN9Jh3IN8GZxwLz-1iyW3s3S8oCa6rnl5ylZryw5PbdKoGZPIaK9EqegiBtqxn0gECkOTifcBeGwJ1JdMje4-HnfPSHANBdfIcxdhCUnv9t9aseqN6byG5uhdZMOFSpdEvtpoNN-xYL24n4oHH3fUPv6Oo0EqumK4StFnhYaZuUkwoAd5_i-4oJI5gF7cJJfEiLmnUysJQqMRi-tgy8j7Ig0A-Um3PJPweDmv80RAa9ZlftQOGZEZkM3mdvja-vwRXZvWxQuGbhPNwRuSDHin_w2t3moABLN5K_qB

\subsection{Uwierzytelnianie użytkownika}

Klient po zalogowaniu się do systemu otrzymuje od serwera dwa tokeny JWT, które są przechowywane w pamięci lokalnej przeglądarki. Aby mógł korzystać z zasobów serwera, musi dołączać pierwszy z nich - token dostępowy - do nagłówka każdego żądania HTTP pod kluczem \texttt{Authorization} poprzedzając go słowem \texttt{Bearer}. Token ten jest ważny przez określony czas, po którym traci ważność - serwer symbolizuje to kodem błędu 401. Jeżeli zaistnieje taka sytuacja, klient musi wysłać żądanie odświeżenia tokena do serwera dołączając drugi token - odświeżający - który ma dłuższy czas przedawnienia. Serwer następnie sprawdza, czy token odświeżający jest poprawny i zwraca nowe tokeny. W przeciwnym wypadku klient musi ponownie zalogować się do systemu. Diagram sekwencji procesu uwierzytelniania użytkownika został przedstawiony na rysunku \ref{fig:authSequence}.

\begin{figure} [H]
    \centering
    \includegraphics[width=0.4\textwidth]{graf/jwtSeq.png}
    \caption{Diagram sekwencji procesu uwierzytelniania użytkownika}
    \label{fig:authSequence}
\end{figure}
% //www.plantuml.com/plantuml/png/fPAnIiH048RxVOhX-a0KAudBaSB2naOG9CraPt8knjcmMGrdAVWKFeQTscdUopLHk1K5zQeKy__pV_zabtr07wukMzN5hpMsGmbmw9q45WBieU5aLAAv-9ZKh5J3cQuPzc5yUhqZ5CjGIM5roUZP0nh3VG_1HOz24-mr1dutOXlWREL8rlCGZavFL5oKwHWOrnrJvmRBD3v2qKGOCAvr_c2nyiooq4MjT_DSiL0qPNgobEDjH4ZbdcaIx-KxbNJ-XWa7iUupLH5lG7rNnj5u7pd6PnQBi-dbOTYiwBdnt9__q379JANRW2VDVtUiIiGDFDlNqxcJyl_-at-Z-1Agr9A5ulDx0m00

\subsection{Harmonogramowanie}

Harmonogramowanie jest częścią systemu, która pochłania najwięcej zasobów serwera, ponieważ dane za każdym razem są przetwarzane na nowo. Odejście od praktyki wstępnego generowania harmonogramów i zapisywania ich w bazie danych umożliwiło zredukowanie ilości danych przechowywanych w bazie oraz konieczności ich aktualizacji w przypadku zmian w grafiku pracy.

\subsubsection{Zapis harmonogramu}

Harmonogramowanie pracy użytkowników odbywa się poprzez wysłanie żądania HTTP z danymi harmonogramu do serwera aplikacyjnego. Struktura wysyłana wraz z żądaniem musi być zgodna z odpowiadającym jej DTO i zawierać informacje o dacie rozpoczęcia, jednostce lub użytkowniku, którego dotyczy oraz blokach czasowych, jakie mają się w~nim znaleźć. Następnie tworzona jest struktura danych odpowiadająca tabelom opisanym w~rozdziale \ref{ss:harmonogramy} i zapisywana w bazie danych. Po zakończeniu procesu, serwer zwraca kod 200 - \texttt{OK}, a aplikacja kliencka przechodzi do widoku harmonogramu. W przypadku błędu, serwer zwraca kod 400 - \texttt{Bad Request}.

\subsubsection{Odczyt harmonogramu}

Odczyt harmonogramu odbywa się poprzez wysłanie żądania \texttt{GET} na adres odpowiednio:
\begin{itemize}
    \item \texttt{/schedule/getUnit/\{id\}/\{startDate\}/\{endDate\}} --- dla harmonogramu jednostki organizacyjnej,
    \item \texttt{/schedule/getUser/\{id\}/\{startDate\}/\{endDate\}} --- dla harmonogramu użytkownika,
\end{itemize}
uzupełniając odpowiednio parametry w nawiasach klamrowych. Oba żądania działają w~podobny sposób - pierwsze z nich uwzględnia wyłącznie harmonogramy jednostki, natomiast drugie łączy harmonogramy wszystkich jednostek przypisanych do użytkownika, a~następnie nadpisuje je harmonogramami użytkownika. W odpowiedzi serwer zwraca dane w formacie \texttt{JSON}, które są przetwarzane przez aplikację kliencką i wyświetlane w~odpowiednim widoku.

Odczyt harmonogramu użytkownika jest dużo bardziej złożony niż jednostki organizacyjnej. Wymaga on pobrania wszystkich jednostek, do których należał użytkownik w~danym okresie czasu, następnie dla każdej z nich odczytania harmonogramu i połączenia ich w jeden widok przechowywany w klasie pomocniczej \texttt{Calendar}. Jeżeli użytkownik posiada również własny harmonogram, serwer generuje jego widok i nadpisuje dane przechowywane w klasie. Diagram sekwencji procesu został przedstawiony na rysunku \ref{fig:scheduleSequence}. \newpage

\begin{figure}[H]
    \centering
    \includegraphics[width=\textwidth]{graf/schSeq.png}
    \caption{Diagram sekwencji odczytu harmonogramu użytkownika}
    \label{fig:scheduleSequence}
\end{figure}
% //www.plantuml.com/plantuml/png/dPJFQXin4CRlynH3xdw1749WxwKWJA6toMf8vDLAYrREOXy3fizG-XYv5T_YVQ-EDhXQkTWizMtdppU_-IJviOyKuhQr05H77x2oXbr4wd7TSm3e96rgqv44xohlOl3MShXB5LMPLVKBwwrbnU7Lr3oLA5NM9D5vVgq0KWnN3rZTuxVT-CkQ3Ox7qq6JCvoep2j5bc5GIPLqtEDN_sGuxD6QFfv-ueQrytta1hVZSHSRFpZJ40xOUH7PjRYd9d1l63N5h2YprmfNdvE_3-7Z_VJZ7qdGF6y0wts7sX8sD1urRywLZG6KtuIePeWl55hl_7EWTfThhx4bYRnn-QdKzAsk8SycVRmF5roQIvrCAfu_i-EmtSXAbfqceNQK-Ff8W-4RmelmXazzFyYwUSGjAcd-Ghep_Hx58yO1avbDRJZtqwL01P80k7a02-z7HbfeDfIB_A-r1Lx1iTJLKg6acZ-bQlLGuL-NSp_Ftb8EjgKid0yfh-TrvsKVFVwUif9EZphJvZpkSVBS01I71sIZ28gvXywCR_WqlfqE-dmdBWNFQG6ya7cKKFex-mC0

\subsection{Przypisanie  karty dostępowej}
\label{sec:addNfc}

Przypisanie karty dostępowej odbywa się przy wykorzystaniu aplikacji mobilnej. Po zalogowaniu się do systemu administrator powinien wybrać zakładkę \texttt{CARDS}, a następnie rozpocząć skanowanie tagu NFC. Po pozytywnym odczytaniu danych aplikacja wyśle do serwera żądanie o informacje na temat karty, a w kolejnym kroku wyświetli informacje o jej właścicielu. Jeżeli karta jest już przypisana do użytkownika, aplikacja umożliwi jej usunięcie, a w przeciwnym wypadku udostępni listę rozwijaną z jego wyborem. Po zatwierdzeniu, aplikacja wyśle do serwera żądanie przypisania karty do użytkownika i~otrzyma odpowiedź. Diagram sekwencji przypisania karty dostępowej do użytkownika został przedstawiony na rysunku \ref{fig:assignCard}.

\begin{figure}[H]
    \centering
    \includegraphics[width=0.7\textwidth]{graf/cardSeq.png}
    \caption{Diagram sekwencji odczytu i przypisania karty dostępowej}
    \label{fig:assignCard}
\end{figure}
% //www.plantuml.com/plantuml/png/TP8nKiCm44LxdM8dVIv0aKaeQ2XIC0mDpThQ38jbIMFBU9oI8OV8v1ZfW2xnlTW8W-r0YgY8tlx_zylpCc0HgjmeJ8ChOA5pje0be5PURZXbZpR0PE4DPvW-uwFDNSB6uYHYte-uQqmpilfqbP1IgASpGU0AxZAqhaRB19dmpScFNp1Gb93VT9QGSEt7SQCZ9cUJFe4xyIbJFo32WavdyAsyrDxLJBfzXtKSobbf6j9H7RMm3qsx4pOOOBjoHIuBaJZKxIkskvJ5nbI3P5ev7-1MyYBuOjruBj6Y0kZtkY-hzgqN88FTVW3zKW9PFcv5Vc3LesHAwbnayGj6or0VziKQ39VXk8Mg_Mn2vchBsM5V2_bFXP5jpj5nZr5-t6BdiJaR_5jgV8DHhVJh6fjRiGb5V7NXTVzaD_t_7KrM3mbHJ8fuFSY0mmIeVn9q3GUK1FP2myD1xwERciif7_uN

\subsection{Uruchomienie czytnika kart}

Uruchomienie czytnika kart odbywa się poprzez podłączenie go do zasilania. Następnie czytnik stara się połączyć z siecią bezprzewodową, która jest zapisana w jego pamięci. Po pozytywnym połączeniu, czytnik wysyła do serwera aplikacyjnego żądanie o informacje na swój temat dołączając do niego adres MAC. Serwer zwraca informacje o czytniku, a ten przechodzi do trybu oczekiwania na odczytanie karty. Jeżeli któryś z kroków nie powiedzie się, czytnik przechodzi w stan błędu i udostępnia sieć Wi-Fi o nazwie \texttt{WMS-AP} z hasłem \texttt{123456789} w celu ręcznego skonfigurowania połączenia. Po połączeniu się innym urządzeniem z siecią czytnika, należy wpisać w przeglądarce adres \texttt{192.168.4.1} i~wprowadzić dane dostępowe do sieci ogólnej. Widok strony konfiguracyjnej czytnika został przedstawiony na rysunku \ref{fig:readerConfig}. Po zatwierdzeniu formularza czytnik zapisuje dane, restartuje się i ponawia opisany wcześniej proces.
\newpage
\section{Połączenie modułów czytnika kart}

Czytnik kart składa się z dwóch modułów oraz kilku elementów pasywnych. Pierwszy z modułów to mikrokontroler \texttt{Raspberry Pi Pico W} odpowiedzialny za przetwarzanie danych i komunikację z serwerem aplikacyjnym. Drugi moduł - oznaczony symbolem \texttt{RFID-RC522} - odpowiada za odczytanie tagów NFC i przekazanie ich do mikrokontrolera. Oba układy są połączone ze sobą za pomocą interfejsu SPI (ang. \english{Serial Peripheral Interface}). Schemat połączeń czytnika kart został przedstawiony na rysunku \ref{fig:readerConnection}.

Do elementów pasywnych układu należą:
\begin{itemize}
    \item Rezystory ograniczające prąd,
    \item Dioda LED RGB (ang. \english{Red, Green, Blue}) ze wspólną katodą, która sygnalizuje stan wewnętrzny czytnika.
\end{itemize}

Zgodnie z sposobem podłączania diod do każdej anody powinien być dołączony rezystor ograniczający prąd. Producent zaleca następujące wartości rezystorów:
\begin{equation}
    \begin{cases}
        R_R=100\Omega \\
        R_G=82\Omega  \\
        R_B=47\Omega
    \end{cases}
\end{equation}
z dokładnością do $5\%$. Niemożność uzyskania rezystora dla kanału zielonego spowodowała, że połączono równolegle dostępne rezystory o wartościach $100\Omega$, $1.2k\Omega$ i $1.2k\Omega$. Dzięki temu uzyskano rezystancję równą:
\begin{equation}
    R_G = \frac{1}{\frac{1}{100}+\frac{1}{1200}+\frac{1}{1200}} \approx 85.71\Omega
\end{equation}
która odbiega od wartości zalecanej o mniej niż $5\%$. \newpage
\begin{figure}[H]
    \centering
    \includegraphics[width=\textwidth]{graf/nfcReader.pdf}
    \caption{Schemat połączeń czytnika kart NFC}
    \label{fig:readerConnection}
\end{figure}
\begin{figure}[H]
    \centering
    \includegraphics[width=0.7\textwidth, frame]{graf/protoCzytnik.jpg}
    \caption{Prototyp czytnika kart na płytce stykowej}
    \label{fig:readerConfig}
\end{figure}