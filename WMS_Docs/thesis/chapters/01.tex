% \begin{itemize}
% \item wprowadzenie w problem/zagadnienie
% \item osadzenie problemu w dziedzinie
% \item cel pracy
% \item zakres pracy
% \item zwięzła charakterystyka rozdziałów
% \end{itemize}

\chapter{Wstęp}
\label{ch:wstep}

\section{Wprowadzenie do problematyki}

W każdym przedsiębiorstwie zatrudniającym pracowników musi zostać ustalona wewnętrzna hierarchia nazywana strukturą organizacyjną. Jest ona oficjalnym podziałem jednostek, komórek, stanowisk i pracowników w organizacji. W nowych przedsiębiorstwach najczęściej występuje nieformalny podział obowiązków, ale z czasem zaczyna klarować się jasny zakres działań poszczególnych osób. Następnie zaczyna się tworzyć struktura organizacyjna, która w pierwszym okresie działania firmy jest spłycona do dwóch poziomów: kierownictwa oraz działów. Taka hierarchia niestety nie sprzyja efektywnemu zarządzaniu, ponieważ obejmuje zbyt szerokie zakresy odpowiedzialności oraz powiązania między działami. \cite{bib:zarzadzanie} W takiej sytuacji konieczne jest wprowadzenie bardziej skomplikowanej struktury organizacyjnej, która umożliwi lepsze zarządzanie. Poniżej wymieniono najczęściej spotykane struktury organizacyjne. \cite{bib:StrukturaOrganizacyjna}

\begin{itemize}
    \item Struktura płaska - występuje w niej centralizacja władzy, w której wszyscy pracownicy podlegają jednemu kierownikowi. Najczęściej występuje w małych lub nowych firmach. Zapewnia szybki przepływ informacji i jest elastyczna.

    \item Struktura liniowa (prosta) - charakteryzuje się występowaniem trzech rodzajów stanowisk: kierowników, pracowników oraz dyrektorów. Każdy pracownik ma przydzielonego jednego przełożonego, który odpowiada przed swoim dyrektorem. W tej strukturze informacje są przekazywane tzw. drogą służbową. Aby pracownik mógł przekazać informacje do dyrekcji, muszą one przejść przez wszystkie szczeble zarządzania. Ten rodzaj struktury jest najbardziej popularny w przedsiębiorstwach franczyzowych.

    \item Struktura funkcjonalna - w tej strukturze każdy z pracowników odpowiada przed kilkoma przełożonymi. Pozwala ona zmniejszyć ilość obowiązków pojedynczego kierownika dzieląc je na kilka osób, które są specjalistami w danej dziedzinie. Każdy specjalista odpowiada przed dyrektorem lub menedżerem, odpowiedzialnym za całość działu.

    \item Struktura sztabowo-liniowa - jest połączeniem struktury liniowej oraz funkcjonalnej. Zespół składa się z jednego kierownika wspieranego przez kilku specjalistów oraz reszty pracowników. Nie istnieje pojedynczy sposób budowy tej struktury, ponieważ każdy z zespołów może mieć inną strukturę wewnętrzną.

    \item Struktura macierzowa (problemowa) - struktura rozdziela kierowników na dwa rodzaje: kierowników projektów oraz kierowników działów. Pracownicy podlegają jednocześnie kilku przełożonym, co zwiększa elastyczność firmy. W zespołach mogą wytworzyć się podgrupy pracowników pracujących nad jednym zadaniem.

    \item Struktura dywizjonalna - struktura określająca dokładną hierarchię w firmie. Najczęściej spotyka się ją w dużych firmach i korporacjach. Charakteryzuje się wyodrębnieniem działów, które są niemal samodzielne. Każdy z działów może mieć własną strukturę organizacyjną, odpowiednią do swoich potrzeb.
\end{itemize}

Struktura organizacyjna, jaką należy przyjąć w danym przedsiębiorstwie zależy ściśle od jego specyfiki oraz wielkości, jednakże w każdym przypadku powinna być ona jasno określona, aby umożliwić jak najskuteczniejsze zarządzanie firmą.

Kolejną krytyczną kwestią w zarządzaniu firmą jest kontrola czasu pracy pracowników. Wpływa ona na ich efektywność pracy i zadowolenie. W zależności od wielkości firmy oraz jej specyfiki, metody kontroli czasu pracy mogą się różnić. W firmach gdzie liczba pracowników jest niewielka, kontrola czasu pracy może być prowadzona w sposób tradycyjny; z kolei duże firmy mogą zastosować bardziej zaawansowane systemy informatyczne. Niezależnie od wybranego podejścia, jej celem jest zapewnienie, aby pracownicy byli obecni w miejscu pracy, w określonym czasie. Prowadzenie kontroli czasu pracy jest obowiązkowe dla pracodawców, a nieprzestrzeganie przepisów jest uznawane za naruszenie praw pracowniczych, co może skutkować karą finansową wysokości zgodnej z art. \nolinebreak 281 pkt \nolinebreak 6 Kodeksu Pracy. \cite{bib:KodeksPracy}

\section{Cel pracy}

% \note{W tej części pracy należy \textbf{jasno} opisać, jaki jest jej cel. Należy zaznaczyć, jakie cele zostaną osiągnięte po zakończeniu pracy.}

Celem pracy jest analiza dziedziny, stwierdzenie zasadności stworzenia systemu, jego projekt oraz implementacja. Docelowo system ma składać się z czterech głównych części:

\begin{itemize}
    \item serwera odpowiadającego za część biznesową,
    \item aplikacji webowej do zarządzania systemem,
    \item systemu mikroprocesorowego do autoryzacji i przyznania dostępu poprzez odczyt kart zbliżeniowych,
    \item aplikacji mobilnej umożliwiającej przypisywanie nowych kart dostępowych.
\end{itemize}

System ma umożliwiać zarządzanie pracownikami, ich zadaniami, czasem pracy oraz strukturą firmy. Dodatkowo ma umożliwiać rejestrację czasu pracy pracowników oraz kontrolę dostępu do pomieszczeń. System ma być kompleksowym narzędziem do zarządzania personelem w firmie, które pozwoli na zautomatyzowanie wielu procesów, zwiększenie efektywności pracy oraz przejrzystość ról w firmie.

\section{Zakres pracy}

W ramach pracy zostanie przeprowadzona analiza dziedziny, na podstawie której zostanie stworzony projekt systemu. Następnie zostaną wyznaczone wymagania funkcjonalne i niefunkcjonalne, które system musi spełniać oraz wykonany zostanie jego projekt graficzny. Po zakończeniu fazy projektowania zostaną rozpoczęte prace nad implementacją kompletnego systemu. Zakończeniu prac będzie towarzyszyło przeprowadzenie testów oraz ewentualne poprawki.

\section{Charakterystyka rozdziałów}

Praca składa się z wielu rozdziałów, z których każdy ma określone zadanie. Poniżej przedstawiono krótką charakterystykę każdego z nich.

\subsection*{Rozdział 2 - Analiza tematu}

W rozdziale zostanie przeprowadzony przegląd dostępnych rozwiązań oraz analiza dziedziny. Zostanie określona zasadność stworzenia systemu oraz przedstawione zostaną najważniejsze funkcjonalności, które powinien on posiadać.

\subsection*{Rozdział 3 - Wymagania i narzędzia}

Rozdział zawiera w sobie wszelkie założenia, jakie musi spełniać system, aby był użyteczny. Zostaną w nim opisane wymagania funkcjonalne i niefunkcjonalne, a także narzędzia i technologie, które zostaną użyte do implementacji systemu.

\subsection*{Rozdział 4 - Specyfikacja zewnętrzna}

Wszelkie informacje dotyczące wyglądu i działania systemu, które są widoczne dla użytkownika, znajdują się w tym rozdziale. Zostaną w nim opisane role użytkowników, dostępne funkcjonalności oraz sposób ich użycia. Znajdzie się w nim również opis interfejsu graficznego oraz obsługi czytnika kart zbliżeniowych.

\subsection*{Rozdział 5 - Specyfikacja wewnętrzna}

Rozdział zawiera w sobie informacje dotyczące implementacji systemu oraz jego architektury. Znajdą się w nim opisy najważniejszych klas, algorytmów, bazy danych i struktury połączeń układu czytnika kart zbliżeniowych. Wyszczególnione zostaną również najważniejsze algorytmy.

\subsection*{Rozdział 6 - Weryfikacja i walidacja}

W rozdziale zostaną opisane testy, które zostały przeprowadzone w celu sprawdzenia poprawności działania systemu oraz sposoby walidacji danych wejściowych.

\subsection*{Rozdział 7 - Podsumowanie i wnioski}

W ostatnim rozdziale pracy zostaną przedstawione wnioski wynikające z przeprowadzonych prac, problemy jakie w ich trakcie wystąpiły oraz propozycje możliwości rozwoju systemu w przyszłości.


