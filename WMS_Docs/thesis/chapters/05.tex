\chapter{Specyfikacja wewnętrzna}
\label{ch:05}

\section{Architektura systemu}

\note{W tej części pracy należy opisać, jakie komponenty składają się na system. Należy zaznaczyć, jakie są relacje między nimi. Można również opisać, jakie wzorce projektowe zostały zastosowane. Głównie chodzi o to, aby czytelnik mógł zrozumieć, jak działa system.}

Kompletny system składa się z czterech głównych komponentów, które komunikują się ze sobą w celu zapewnienia pełnej funkcjonalności. Poniżej zostały przedstawione opisy każdego z nich.

\subsection{Frontend}

Głównym interfejsem użytkownika jest aplikacja webowa stworzona przy użyciu frameworka \texttt{Vue.js}. Zapewnia użytkownikowi dostęp do większości funkcji systemu, a w zależności od roli użytkownika, umożliwia wykonanie innych czynności. Aplikacja została zaprojektowana w taki sposób, aby była przejrzysta i intuicyjna w obsłudze. Dzięki temu, użytkownik może szybko i sprawnie wykonywać swoje codzienne obowiązki. Odrębne od siebie panele zostały umieszczone w kartach pojawiających się na widoku, dzięki czemu użytkownik wie dokładnie w którym miejscu aplikacji się znajduje i jakie ma możliwości. Frontend komunikuje się z serwerem aplikacyjnym wysyłając żądania HTTP i odbierając odpowiedzi w formacie \texttt{JSON}.

\subsection{Backend}

Serwer aplikacyjny został stworzony przy użyciu frameworka \texttt{Spring Boot}. Zapewnia on komunikację między bazą danych, a pozostałymi komponentami systemu. Jest odpowiedzialny za przetwarzanie żądań klienckich oraz zwracanie odpowiedzi w postaci danych, w formacie \texttt{JSON}. Serwer aplikacyjny jest również odpowiedzialny za autoryzację i autentykację użytkowników oraz zarządzanie sesjami. Użytkownicy nie mają bezpośredniego dostępu do serwera.

\subsection{Aplikacja mobilna}

Aplikacja mobilna umożliwia użytkownikom przeglądanie swoich danych bez konieczności korzystania z przeglądarki internetowej. Jej głównym zadaniem jest jednak możliwość dodawania kart dostępowych dla poszczególnych użytkowników. Odbywa się to poprzez przyłożenie tagu NFC (ang. \english{Near Field Communication}) do telefonu z zainstalowaną aplikacją, a następnie przypisanie go do konkretnego użytkownika. Szczegółowy opis tego procesu został opisany w rozdziale \ref{sec:addNfc}.

\subsection{Układ mikroprocesorowy}

Układ mikroprocesorowy jest odpowiedzialny za odczytywanie tagów NFC oraz przesyłanie informacji do serwera aplikacyjnego. Następnie serwer zwraca informację o przyznaniu dostępu do systemu. Układ mikroprocesorowy jest zasilany z wbudowanego portu Micro USB, co umożliwia bardzo proste podłączenie go do źródła zasilania. Dokumentacja techniczna mikrokontrolera \cite{bib:picoWdatasheet} precyzuje również podłączenie innego źródła zasilania bez użycia wbudowanego portu.

\section{Struktura bazy danych}

\note{W tej części pracy należy opisać, jakie tabele i relacje między nimi występują w bazie danych. Można dodać diagramy ERD.}

Baza danych obsługująca system składa się z kilkunastu tabel, które przechowują informacje o wszelkich danych w systemie. Na rysunku \ref{fig:dbDiagram} został przedstawiony jej uproszczony schemat, a w kolejnych podrozdziałach zostaną opisane najważniejsze tabele oraz ich relacje.

\begin{figure}[H]
    \centering
    \includegraphics[width=\textwidth]{graf/dbDiagram.png}
    \caption{Uproszczony schemat bazy danych}
    \label{fig:dbDiagram}
\end{figure}

\subsection{Tabele użytkowników}

\begin{figure}[H]
    \centering
    \includegraphics[width=0.7\textwidth]{graf/usersTable.png}
    \caption{Schemat tabel użytkowników}
    \label{fig:usersTable}
\end{figure}

Główną tabelą przechowującą informacje o użytkownikach jest tabela \texttt{USERS}. Zawiera ona dane personalne, takie jak imię, nazwisko, adres e-mail i numer telefonu. Dodatkowo do tabeli wpisane są dane do logowania: nazwa użytkownika i hasło; oraz informacja o użytkowniku, który  utworzył dany wpis. Każdy użytkownik ma przypisaną jedną z ról z tabeli \texttt{DICT\_AUTHORITIES}, która określa jego uprawnienia w systemie - w relacji wiele do jednego. Relacją jeden do jednego jest połączona tabela \texttt{USER\_PAYOFF\_DATA} zawierające dane o koncie bankowym użytkownika. W tabeli \texttt{USERS} znajduje się również pole \texttt{is\_archived}, które określa, czy użytkownik jest aktywny w systemie.

\subsection{Tabele jednostek organizacyjnych}

\begin{figure}[H]
    \centering
    \includegraphics[width=0.8\textwidth]{graf/unitsTable.png}
    \caption{Schemat tabel jednostek organizacyjnych}
    \label{fig:organizationalUnitsTable}
\end{figure}

Wszystkie jednostki organizacyjne są przechowywane w tabeli \texttt{UNITS} zawierającej dane o nazwie jednostki, jej opisie, jednostce nadrzędnej oraz dacie utworzenia. Pole \texttt{work\_ended} określa, czy jednostka jest aktywna. Użytkownicy są przypisani do jednostki organizacyjnej poprzez tabelę \texttt{positions} będącą tabelą łącznikową między tabelami \texttt{USERS} i \texttt{UNITS}. Znajdują się w niej dane o dacie rozpoczęcia pracy na danym stanowisku oraz dacie zakończenia pracy, a także odwołanie do tabeli słownikowej, zawierającej nazwy stanowisk. Tabela \texttt{POSITIONS} jest szczególnie ważna przy odczytywaniu harmonogramów pracy.

\subsection{Tabele harmonogramów}

\begin{figure}[H]
    \centering
    \includegraphics[width=0.7\textwidth]{graf/scheduleTable.png}
    \caption{Schemat tabel harmonogramów}
    \label{fig:schedulesTable}
\end{figure}

Na każdy z harmonogramów składa się pojedynczy rekord w tabeli \texttt{SCHEDULES} oraz pewna liczba rekordów w tabeli \texttt{SCHEDULE\_BLOCKS}. Pierwsza z nich zawiera informacje o dacie rozpoczęcia i zakończenia harmonogramu, jego utworzenia oraz odwołanie do jednostki organizacyjnej lub użytkownika, którego dotyczy. Tabela \texttt{SCHEDULE\_BLOCKS} odpowiada za przechowywanie pojedynczych bloków czasowych w harmonogramie opisując ich dzień i godzinę rozpoczęcia oraz zakończenia oraz jednostkę, której dotyczą. Tabele są powiazane relacją jeden do wielu - jeden harmonogram może zawierać wiele bloków czasowych.

\subsection{Tabele kart dostępowych}

\begin{figure}[H]
    \centering
    \includegraphics[width=0.7\textwidth]{graf/acTable.png}
    \caption{Schemat tabel kart dostępowych}
    \label{fig:accessCardsTable}
\end{figure}

Karty dostępowe użytkowników są przechowywane w tabeli \texttt{ACCESS\_CARDS} zawierającej informacje o numerze seryjnym, jej właścicielu opisie, typie karty oraz statusie. Dzięki ostatniej z właściwości możliwe jest przypisanie dwóm użytkownikom jednej karty w różnych okresach czasu. Każda autoryzacja użytkownika jest zapisywana w tabeli \texttt{USER\_ACCESS} wpisując do niej datę i godzinę, id użytkownika, id czytnika, id karty dostępowej oraz status autoryzacji. Umożliwia to późniejsze analizowanie historii autoryzacji.

Tabela \texttt{DEVICES} przechowuje informacje o czytnikach kart - ich opis, datę ostatniego uruchomienia, adres MAC (ang. \english{Media Access Control address}) oraz symbol, którym identyfikuje się w systemie.

\subsection{Tabele słownikowe}

W bazie danych znajdują się trzy tabele słownikowe zawierające dane, które nie zmieniają się w czasie działania systemu. Każda z nich poprzedzona jest prefiksem \texttt{DICT\_}. Są to:

\begin{itemize}
    \item \texttt{DICT\_AUTHORITIES} zawierająca role użytkowników, równoznaczne z uprawnieniami,
    \item \texttt{DICT\_ACCESS\_CARD\_TYPES} zawierająca typy kart dostępowych dla łatwiejszego rozróżnienia tagów NFC,
    \item \texttt{DICT\_POSITION\_NAMES} zawierająca nazwy stanowisk przypisanych pracownikom.
\end{itemize}

Do tabel \texttt{DICT\_ACCESS\_CARD\_TYPES} i \texttt{DICT\_POSITION\_NAMES} mogą zostać dodane nowe rekordy, lecz nie jest możliwe ich usunięcie. Takie ograniczenie zapewnia integralność danych w systemie i zapobiega błędom w działaniu aplikacji.

Dane mogą dodawać jedynie administratorzy systemu.

\section{Modele i struktury danych}

\note{W tej części pracy należy opisać, jakie modele danych zostały zastosowane w systemie. Należy zaznaczyć, jakie są relacje między nimi.}

\subsection{Model użytkownika}

\section{Algorytmy}

\subsection{Rejestracja i pierwsze logowanie użytkownika}

Rejestracja użytkownika może zostać dokonana jedynie przez administratora systemu. W tym celu powinien on wypełnić formularz rejestracyjny wprowadzając nazwę użytkownika oraz jego adres email. Po zatwierdzeniu formularza, w widoku wszystkich użytkowników pojawi się nowy rekord z danymi właśnie utworzonego użytkownika. Następnie administrator powinien przekazać nowo zarejestrowanemu użytkownikowi jego nazwę, którą musi wpisać na ekranie logowania - nie jest wymagane przy tym wpisywanie hasła. Po zatwierdzeniu formularza, i wysłaniu danych do serwera, sprawdzi on czy użytkownik o podanej nazwie istnieje w bazie danych i czy ma przypisane hasło. Jeżeli nie, zwróci kod 206 - \texttt{Partial Content}, a aplikacja udostępni użytkownikowi możliwość ustawienia hasła. Po jego wpisaniu, potwierdzeniu i zatwierdzeniu formularza, użytkownik zostanie zalogowany do systemu. Diagram sekwencji pierwszego logowania użytkownika został przedstawiony na rysunku \ref{fig:login}, a szczegóły tego procesu zostały opisane w rozdziale \ref{ss:logowanie}

\subsection{Logowanie}
\label{ss:logowanie}

Logowanie do systemu odbywa się poprzez przesłanie żądania HTTP z danymi logowania do serwera aplikacyjnego. Po jego otrzymaniu serwer zwraca się do bazy danych w celu znalezienia użytkownika o podanej nazwie. Jeżeli użytkownik nie istnieje, serwer zwraca kod błędu 401. W przeciwnym wypadku sprawdza, czy zahashowane hasło użytkownika jest zgodne z zapisanym w bazie. Jeżeli hasła się zgadzają, serwer generuje dwa tokeny JWT (ang. \english{JSON Web Token}) i zwraca je w odpowiedzi. Aplikacja kliencka zapisuje otrzymane tokeny w pamięci lokalnej przeglądarki i przechodzi do widoku panelu głównego. Diagram czynności logowania został przedstawiony na rysunku \ref{fig:login}.

\begin{figure}[H]
    \centering
    \includegraphics[width=0.9\textwidth]{graf/loginSequence.png}
    \caption{Diagram czynności logowania}
    \label{fig:login}
\end{figure}
% //www.pantum.com/plantuml/png/VP8nRiCm34LtdKB8dWju288CdOgYIz2PigX6iIC5jbJN7WFq43rBEzRtAeceK6pKcJpyh__9Hs_R04s8frf06NmZz-Dt7ohVELj9QAMBuaowBUqPN90FZNS1dMR9u4JQGLabHQ7G4411YtArWm6a1jUNXnMBMWaXN9Jh3IN8GZxwLz-1iyW3s3S8oCa6rnl5ylZryw5PbdKoGZPIaK9EqegiBtqxn0gECkOTifcBeGwJ1JdMje4-HnfPSHANBdfIcxdhCUnv9t9aseqN6byG5uhdZMOFSpdEvtpoNN-xYL24n4oHH3fUPv6Oo0EqumK4StFnhYaZuUkwoAd5_i-4oJI5gF7cJJfEiLmnUysJQqMRi-tgy8j7Ig0A-Um3PJPweDmv80RAa9ZlftQOGZEZkM3mdvja-vwRXZvWxQuGbhPNwRuSDHin_w2t3moABLN5K_qB

\subsection{Uwierzytelnianie użytkownika}

Klient po zalogowaniu się do systemu otrzymuje od serwera dwa tokeny JWT, które są przechowywane w pamięci lokalnej przeglądarki. Aby mógł korzystać z zasobów serwera, musi dołączać pierwszy z nich - token dostępowy - do nagłówka każdego żądania HTTP pod kluczem \texttt{Authorization} poprzedzając go słowem \texttt{Bearer}. Przykładowy nagłówek został przedstawiony na rysunku. Token ten jest ważny przez określony czas, po którym traci ważność - serwer zwraca wtedy kod błędu 401. Jeżeli zaistnieje taka sytuacja, klient musi wysłać żądanie odświeżenia tokena do serwera, dołączając drugi token - odświeżający - który ma dłuższy czas przedawnienia. Serwer następnie sprawdza, czy token odświeżający jest poprawny i zwraca nowe tokeny. W przeciwnym wypadku klient musi ponownie zalogować się do systemu. Diagram sekwencji procesu uwierzytelniania użytkownika został przedstawiony na rysunku \ref{fig:authSequence}.

\begin{figure} [H]
    \centering
    \includegraphics[width=0.4\textwidth]{graf/jwtSeq.png}
    \caption{Diagram sekwencji procesu uwierzytelniania użytkownika}
    \label{fig:authSequence}
\end{figure}
% //www.plantuml.com/plantuml/png/fPAnIiH048RxVOhX-a0KAudBaSB2naOG9CraPt8knjcmMGrdAVWKFeQTscdUopLHk1K5zQeKy__pV_zabtr07wukMzN5hpMsGmbmw9q45WBieU5aLAAv-9ZKh5J3cQuPzc5yUhqZ5CjGIM5roUZP0nh3VG_1HOz24-mr1dutOXlWREL8rlCGZavFL5oKwHWOrnrJvmRBD3v2qKGOCAvr_c2nyiooq4MjT_DSiL0qPNgobEDjH4ZbdcaIx-KxbNJ-XWa7iUupLH5lG7rNnj5u7pd6PnQBi-dbOTYiwBdnt9__q379JANRW2VDVtUiIiGDFDlNqxcJyl_-at-Z-1Agr9A5ulDx0m00

\subsection{Harmonogramowanie}

\subsection{Przypisanie  karty dostępowej}

Przypisanie karty dostępowej odbywa się przy wykorzystaniu aplikacji mobilnej. Po zalogowaniu się do systemu administrator powinien wybrać zakładkę \texttt{CARDS}, a następnie rozpocząć skanowanie tagu NFC. Po pozytywnym odczytaniu danych aplikacja wyśle do serwera żądanie o informacje na temat karty, a w kolejnym kroku wyświetli informacje jej właścicielu. Jeżeli karta jest już przypisana do użytkownika, aplikacja umożliwi jej usunięcie, a w przeciwnym wypadku udostępni menu wyboru użytkownika, do którego ma zostać przypisana. Po zatwierdzeniu wyboru, aplikacja wyśle do serwera żądanie przypisania karty do użytkownika. Diagram sekwencji przypisania karty dostępowej został przedstawiony na rysunku \ref{fig:assignCard}.


\begin{figure}[H]
    \centering
    \includegraphics[width=0.7\textwidth]{graf/cardSeq.png}
    \caption{Diagram sekwencji odczytu i przypisania karty dostępowej}
    \label{fig:assignCard}
\end{figure}
% //www.plantuml.com/plantuml/png/TP8nKiCm44LxdM8dVIv0aKaeQ2XIC0mDpThQ38jbIMFBU9oI8OV8v1ZfW2xnlTW8W-r0YgY8tlx_zylpCc0HgjmeJ8ChOA5pje0be5PURZXbZpR0PE4DPvW-uwFDNSB6uYHYte-uQqmpilfqbP1IgASpGU0AxZAqhaRB19dmpScFNp1Gb93VT9QGSEt7SQCZ9cUJFe4xyIbJFo32WavdyAsyrDxLJBfzXtKSobbf6j9H7RMm3qsx4pOOOBjoHIuBaJZKxIkskvJ5nbI3P5ev7-1MyYBuOjruBj6Y0kZtkY-hzgqN88FTVW3zKW9PFcv5Vc3LesHAwbnayGj6or0VziKQ39VXk8Mg_Mn2vchBsM5V2_bFXP5jpj5nZr5-t6BdiJaR_5jgV8DHhVJh6fjRiGb5V7NXTVzaD_t_7KrM3mbHJ8fuFSY0mmIeVn9q3GUK1FP2myD1xwERciif7_uN

\section{Użyte biblioteki i frameworki}

\note{Ta część może się znaleźć przy przeglądzie technologii.}