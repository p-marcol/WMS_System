% \begin{itemize}
% \item sformułowanie problemu
% \item osadzenie tematu w kontekście aktualnego stanu wiedzy (\english{state of the art}) o poruszanym problemie
% \item  studia literaturowe \cite{bib:artykul,bib:ksiazka,bib:konferencja,bib:internet} -  opis znanych rozwiązań (także opisanych naukowo, jeżeli problem jest poruszany w publikacjach naukowych), algorytmów, 
% \end{itemize}


% Wzory  
% \begin{align}
% y = \frac{\partial x}{\partial t}
% \end{align}
% jak i pojedyncze symbole $x$ i $y$  składa się w trybie matematycznym.


%%%%%%%%%%%%%%%%%%%%%%%%


\chapter{Analiza tematu}

% \section{Kontrola czasu pracy w historii}

% \note{W tej części można opisać jak wyglądała kontrola czasu pracy w przeszłości. Należy zaznaczyć, jakie metody były stosowane, jakie były ich wady i zalety. Można również opisać, jakie były konsekwencje nieprzestrzegania zasad kontroli czasu pracy.}

\section{Przegląd dostępnych sposobów kontroli czasu}

\subsection{Metody tradycyjne}

Wiele małych przedsiębiorstw nie potrzebuje wprowadzenia zaawansowanych systemów kontroli pracowników i wciąż korzysta z metod tradycyjnych. Polegają one w większości na ręcznym wypełnianiu papierowych dokumentów, które następnie wymagają przetworzenia. Przykłady takich metod wymieniono poniżej.

\begin{itemize}
    \item \textbf{Indywidualne karty czasu pracy} \cite{bib:PotwierdzenieObecnosci} - pracownik zaznacza swoją obecność na~kartce papieru, a następnie podpisuje ją. Zaletą tej metody jest jej prostota i niski koszt, jednakże jest ona mało efektywna i podatna na błędy. Dodatkowo wymagane jest przechowywanie dużej ilości papierowych dokumentów oraz ich archiwizacji.
    \item \textbf{Arkusz kalkulacyjny} - metoda polegająca na prowadzeniu arkusza kalkulacyjnego, w którym - podobnie jak w przypadku kart czasu - zapisywane są godziny pracy pracowników oraz zadania jakie w tym czasie wykonali. Arkusze pozwalają na jednoczesną kontrolę obecności pracowników oraz na monitorowanie ich postępów w pracy. Niestety, brak automatyzacji procesu skutkuje koniecznością ręcznego wypełniania arkuszy, co zwiększa ryzyko popełnienia błędów. Można wykorzystywać do tego programy takie jak Microsoft Excel czy Google Sheets \cite{bib:ExcelTimeTracking}.
    \item \textbf{Harmonogram pracy} - pozwala przełożonym na ustalenie grafiku pracy pracowników. Jest on następnie przekazywany podwładnym, którzy muszą go przestrzegać. Ta metoda wymaga wykorzystania dodatkowych narzędzi, np. kart pracy. Jej największą wadą jest brak możliwości przekazania informacji o zmianach w grafiku, w~czasie rzeczywistym, co może prowadzić do nieporozumień i konfliktów.
\end{itemize}

Połączenie kilku tradycyjnych metod pozwala na skuteczną kontrolę czasu pracy, lecz bez wykorzystania systemów informatycznych, proces ten jest bardzo czasochłonny i podatny na błędy. Korzystanie z papierowych dokumentów jest obarczone bardzo dużym ryzykiem utraty danych, nieautoryzowanego dostępu oraz błędów ludzkich \cite{bib:LuceosSmart}.

\subsection{Systemy zarządzania kapitałem ludzkim}

Wraz z rozwojem technologii informatycznych wiele firm zdecydowało się na wprowadzenie rozwiązań HCM (ang. \english{Human Capital Management}). Pozwalają one na pełną automatyzację procesów związanych z działaniem kadr i płac. Oferują szereg funkcji związanych z zarządzaniem czasem pracy, rekrutacją, szkoleniami, wynagrodzeniami oraz rozwojem pracowników. Ich główną częścią jest moduł \english{Workforce Management} \cite{bib:OracleHCMWhatIs}. Zazwyczaj dostęp do systemu odbywa się poprzez aplikację internetową, co umożliwia łatwy dostęp z dowolnego miejsca na świecie. Przykłady dostępnych systemów HCM wymieniono poniżej.

\begin{itemize}
    \item \textbf{Oracle HCM Cloud} \cite{bib:OracleHCM} - kompletne rozwiązanie chmurowe firmy Oracle, które łączy w sobie funkcje zarządzania personelem, procesami kadrowymi, rekrutacyjnymi i płacowymi. Jest używany m. in. przez FUJIFILM, Deutsche Bahn, Fujitsu.
    \item \textbf{SAP SuccessFactors HCM} \cite{bib:SAPHCM} - rozwiązanie chmurowe firmy SAP, które oferuje szereg funkcji w zakresie HR (ang. \english{Human Resources}). Zawiera w sobie moduły do zarządzania procesami kadrowymi, rekrutacyjnymi, szkoleniowymi, płacowymi i~analitycznymi. Jest używany m. in. przez Microsoft, Nestle, Allianz.
    \item \textbf{MintHCM} \cite{bib:MintHCM} - oprogramowanie firmy eVolpe oparte o otwartoźródłowe systemy CRM (ang. \english{Customer Relationship Management}). Oferuje szereg funkcji związanych z zarządzaniem personelem, takich jak: rekrutacja, szkolenia, oceny pracownicze czy zarządzanie czasem pracy i urlopami.  Korzystają z niego m. in. Empik, Poczta Polska, Asseco.
\end{itemize}

Głównym powodem decyzji firm o wdrożeniu systemów zarządzania kapitałem ludzkim jest ich pozytywny wpływ na efektywność pracy i wzrost przychodów. Sprawnie działający system umożliwiający załatwienie wielu spraw w jednym miejscu jest ogromnym udogodnieniem dla pracowników, pozwalając na szybsze reagowanie na zmiany w organizacji i~dając jasny wgląd do danych dotyczących ich wydajności. Dla kadry zarządzającej użycie takich systemów daje wymierne korzyści, umożliwiając bardzo szybkie przeglądanie statystyk pracowników i generowanie raportów \cite{bib:ZarzadzanieZasobamiLudzkimi}. Niestety, ich wdrożenie wiąże się najczęściej z ogromnymi kosztami oraz koniecznością przeprowadzenia szkoleń pracowników, co może być problematyczne dla małych przedsiębiorstw.

Często w rozwiązaniach HCM brakuje funkcji związanej z przyznawaniem dostępów oraz kontrolą wejść i wyjść pracowników. W takich przypadkach konieczne jest zintegrowanie ich z systemem kontroli dostępu, co może zwiększać koszta i skomplikowanie całości. Dodatkowo, systemy HCM są zazwyczaj dostępne jedynie w formie chmurowej, co może okazać się problemem dla firm, które chcą mieć pełną kontrolę nad danymi swoich pracowników.

\subsection{Systemy kontroli dostępu}

Celem systemów kontroli dostępu jest zapewnienie bezpieczeństwa w firmie poprzez kontrolę wejść i wyjść pracowników oraz gości. Zapewniają one bezpieczeństwo pracownikom oraz ochronę mienia firmy. Pozwalają również na identyfikację osób przemieszczających się po budynku oraz na ograniczenie dostępu do poszczególnych pomieszczeń. Zdarza się, że systemy są zintegrowane z alarmami oraz monitoringiem. Zazwyczaj spotyka się je w dużych firmach, gdzie kontrola dostępu jest kluczowym elementem bezpieczeństwa~\cite{bib:KontrolaDostepu}. Takie systemy dostarczają m. in. firmy:

\begin{itemize}
    \item \textbf{Satel} \cite{bib:satel} - polska firma zajmująca się produkcją systemów alarmowych, monitoringowych i kontroli dostępu, której rozwiązania opierają się o technologię RFID (ang. \english{Radio-Frequency Identification}). Możliwe jest wdrożenie lokalne lub rozproszone,
    \item \textbf{Avigilon} \cite{bib:avigilon} - firma zajmująca się produkcją systemów monitoringu i kontroli dostępu. Ich rozwiązania opierają się głównie na technologiach bezprzewodowych oraz~pinpadach.
\end{itemize}

Systemy kontroli dostępu są zazwyczaj stosowane w firmach, w których bezpieczeństwo jest kluczowym elementem. Dla pracowników ich użytkowanie jest proste i intuicyjne, a~dostęp do poszczególnych pomieszczeń jest szybki i wygodny. Niestety, systemy te nie oferują funkcji związanych z zarządzaniem personelem i śledzeniem czasu pracy. W takich przypadkach konieczne jest zintegrowanie ich z systemem HCM, co stanowczo zwiększa koszty i skomplikowanie systemu.

\subsection{Podsumowanie}

Analiza dziedziny pozwala na stwierdzenia, że istnieje zapotrzebowanie na system łączący w sobie funkcje zarządzania personelem, kontroli czasu pracy oraz kontroli dostępu. Obecne rozwiązania są albo nieefektywne i przestarzałe, albo nie zawierają w sobie wszystkich funkcjonalności. W związku z tym zaprojektowanie i wdrożenie nowego systemu, który pozwoli na automatyzację wymienionych procesów, może przynieść wymierne korzyści dla firm. Taki system pozwoli na zwiększenie efektywności pracy, bezpieczeństwa oraz ułatwi pracownikom codzienną pracę. Dodatkowo, wykluczy on koszty ponoszone na~utrzymanie kilku systemów oraz zintegrowanie ich ze sobą.

% \note{tu może być jeszcze dopisana dywagacja na temat zasadności wdrożenia systemu zarządzania personelem}

\section{Najważniejsze funkcjonalności systemu}

% \note{W tej części pracy można opisać, jakie funkcjonalności powinien posiadać system zarządzania personelem. Należy zaznaczyć, jakie są one najważniejsze, a także jakie korzyści mogą przynieść firmie.}

Dobrze zaprojektowany system zarządzania personelem powinien zawierać szereg modułów związanych z jego kluczowymi częściami. Poniżej wymieniono najważniejsze z nich.

\begin{itemize}
    \item \textbf{Zarządzanie pracownikami} - pozwala na przechowywanie danych pracowników i~zarządzanie nimi. W systemie powinna być możliwość dodawania, edycji oraz usuwania pracowników, a także przypisywania im odpowiednich ról i uprawnień.
    \item \textbf{Zarządzanie czasem pracy} - pozwala na kontrolę czasu pracy pracowników. System powinien umożliwiać zarządzanie grafikami pracy, kontrolę obecności pracowników oraz monitorowanie ich postępów w pracy.
    \item \textbf{Zarządzanie dostępem} - umożliwia na kontrolę dostępu osób do budynku. Moduł powinien umożliwiać zarządzanie kartami dostępowymi, kontrolować wejścia i~wyjścia pracowników, a także umożliwiać nadawanie uprawnień dostępu do poszczególnych pomieszczeń.
    \item \textbf{Zarządzanie strukturą firmy} - udostępnia możliwość zarządzania strukturą wewnętrzną przedsiębiorstwa. System powinien pozwalać na tworzenie i zarządzanie działami, zespołami oraz stanowiskami pracy.
    \item \textbf{Zarządzanie wnioskami} - pracownicy powinni mieć możliwość składania wszelkich wniosków w systemie. Moduł powinien umożliwiać zarządzanie wnioskami urlopowymi, zwolnieniami lekarskimi, czy wnioskami o zmianę grafiku pracy.
\end{itemize}

