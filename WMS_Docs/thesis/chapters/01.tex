% \begin{itemize}
% \item wprowadzenie w problem/zagadnienie
% \item osadzenie problemu w dziedzinie
% \item cel pracy
% \item zakres pracy
% \item zwięzła charakterystyka rozdziałów
% \end{itemize}

\chapter{Wstęp}
\label{ch:wstep}

\section{Wprowadzenie do problematyki}

% \note{W tej części pracy należy wprowadzić czytelnika w tematykę pracy. Należy opisać problem, który zostanie rozwiązany w pracy, a także zaznaczyć, dlaczego jest on ważny.}

W każdym przedsiębiorstwie zatrudniającym pracowników musi zostać ustalona wewnętrzna hierarchia nazywana strukturą organizacyjną. Jest ona oficjalnym podziałem jednostek, komórek, stanowisk i pracowników w organizacji. W nowych przedsiębiorstwach często występuje nieformalny podział obowiązków, ale z czasem zaczyna klarować się jasny zakres działań poszczególnych osób. Następnie zaczyna się tworzyć struktura organizacyjna, która w pierwszym okresie działania firmy jest spłycona do dwóch poziomów: kierownictwa oraz działów. Niestety, taka hierarchia uniemożliwia skuteczne zarządzanie, ponieważ występują w niej zbyt duże obszary odpowiedzialności i skorelowanie działów. \cite{bib:zarzadzanie} W takiej sytuacji konieczne jest wprowadzenie bardziej skomplikowanej struktury organizacyjnej, która pozwoli na lepsze zarządzanie firmą. Poniżej wymieniono najczęściej spotykane struktury organizacyjne. \cite{bib:StrukturaOrganizacyjna}

\begin{itemize}
    \item Struktura płaska - występuje w niej centralizacja władzy, w której wszyscy pracownicy podlegają jednemu kierownikowi. Najczęściej występuje w małych lub nowych firmach. Zapewnia szybki przepływ informacji i jest elastyczna.

    \item Struktura liniowa (prosta) - charakteryzuje się występowaniem trzech rodzajów stanowisk: kierowników, pracowników oraz dyrektorów. Każdy pracownik ma przydzielonego jednego przełożonego, który odpowiada przed swoim dyrektorem. W tej strukturze informacje są przekazywane tzw. drogą służbową. Aby pracownik mógł przekazać informacje do dyrekcji, muszą one przejść przez wszystkie szczeble zarządzania. Ten rodzaj struktury jest najbardziej popularny w przedsiębiorstwach franczyzowych.

    \item Struktura funkcjonalna - w tej strukturze każdy z pracowników odpowiada przed kilkoma przełożonymi. Pozwala ona zmniejszyć ilość obowiązków pojedynczego kierownika dzieląc je na kilka osób, które są specjalistami w danej dziedzinie. Każdy specjalista odpowiada przed dyrektorem lub menedżerem, który jest odpowiedzialny za całość działu.

    \item Struktura sztabowo-liniowa - jest połączeniem struktury liniowej oraz funkcjonalnej. Zespół składa się z jednego kierownika wspieranego przez kilku specjalistów oraz pracowników. Nie istnieje pojedynczy sposób budowy tej struktury, ponieważ każdy z zespołów może mieć inną strukturę wewnętrzną.

    \item Struktura macierzowa (problemowa) - struktura rozdziela kierowników na dwa rodzaje: kierowników projektów oraz kierowników działów. Pracownicy podlegają jednocześnie kilku przełożonym, co zwiększa elastyczność firmy. W zespołach mogą wytworzyć się podgrupy pracowników pracujących nad jednym zadaniem.

    \item Struktura dywizjonalna - struktura określająca dokładną hierarchię w firmie. Najczęściej spotyka się ją w dużych firmach i korporacjach. Charakteryzuje się wyodrębnieniem działów, które są niemal samodzielne. Każdy z działów może mieć własną strukturę organizacyjną, odpowiednią do swoich potrzeb.
\end{itemize}

Struktura organizacyjna, jaką należy przyjąć w danym przedsiębiorstwie zależy ściśle od jego specyfiki oraz wielkości, jednakże w każdym przypadku powinna być ona jasno określona, aby umożliwić skuteczne zarządzanie firmą.

Kolejną krytyczną kwestią w zarządzaniu firmą jest kontrola czasu pracy pracowników. Wpływa ona na efektywność pracy, a także na zadowolenie pracowników. W zależności od wielkości firmy oraz od jej specyfiki, metody kontroli czasu pracy mogą się różnić. W firmach, gdzie liczba pracowników jest niewielka, kontrola czasu pracy może być prowadzona w sposób tradycyjny; z kolei duże firmy mogą zastosować bardziej zaawansowane systemy informatyczne. Niezależnie od wybranego podejścia, celem jest zapewnienie, aby pracownicy byli obecni w miejscu pracy, w określonym czasie. Prowadzenie kontroli czasu pracy jest obowiązkowe dla pracodawców, a nieprzestrzeganie przepisów jest uznawane za naruszenie praw pracowniczych, co może skutkować karą finansową wysokości zgodnej z art. \nolinebreak 281 pkt \nolinebreak 6 Kodeksu Pracy. \cite{bib:KodeksPracy}

\section{Cel pracy}

\note{W tej części pracy należy \textbf{jasno} opisać, jaki jest jej cel. Należy zaznaczyć, jakie cele zostaną osiągnięte po zakończeniu pracy.}

Celem pracy jest analiza dziedziny, stwierdzenie zasadności wprowadzenia systemu, jego projekt oraz implementacja. Docelowo system ma składać się z czterech głównych części:

\begin{itemize}
    \item serwera odpowiadającego za część biznesową,
    \item aplikacji webowej do zarządzania systemem,
    \item systemu mikroprocesorowego do autoryzacji i przyznania dostępu poprzez odczyt kart zbliżeniowych,
    \item aplikacji mobilnej umożliwiającej przypisywanie nowych kart dostępowych.
\end{itemize}

\section{Zakres pracy}

\note{W tej części pracy należy opisać, jakie zagadnienia zostaną poruszone w pracy. Należy zaznaczyć, jakie aspekty problemu zostaną omówione, a jakie nie.}

\section{Charakterystyka rozdziałów}

\note{\textbf{TODO}: W tej części pracy należy zwięźle opisać, co znajduje się w poszczególnych rozdziałach.}
