\chapter{Weryfikacja i walidacja}
\label{ch:06}

\section{Walidacja danych}

Walidacja danych przesyłanych pomiędzy serwerem, a klientem jest jednym z kluczowych elementów w procesie tworzenia aplikacji internetowych. Jej brak może doprowadzić do poważnych problemów związanych z bezpieczeństwem. Szczególnie ważne jest to w przypadku aplikacji, które wymagają od użytkownika wprowadzenia danych.

\subsection{Walidacja po stronie klienta}

Walidacja po stronie klienta jest realizowana przy pomocy biblioteki \texttt{Valibot}. Opiera się ona na obiektach walidujących, którymi sprawdzany jest formularz przed jego wysłaniem. Tworzenie obiektów jest bardzo proste i intuicyjne - polega na tworzeniu potoków, które zawierają kolejne funkcje walidujące. Przykład walidacji został przedstawiony na listingu \ref{lst:valibot}.

\begin{listing}[H]
    \begin{minted}[linenos, breaklines]{js}
const newRecordSchema = v.object({
    unitId: v.required(v.number()),
    hours: v.required(v.pipe(v.number(), v.minValue(0.25), v.maxValue(24))),
    date: v.required(v.date()),
    description: v.required(v.pipe(v.string(), v.nonEmpty(), v.maxLength(255))),
})

const result = v.safeParse(newRecordSchema, requestBody);
    \end{minted}
    \caption{Przykład walidacji formularza przy pomocy biblioteki Valibot}
    \label{lst:valibot}
\end{listing}

\subsection{Walidacja po stronie serwera}

Do walidacji danych wykorzystano rozwiązanie udostępniane przez \texttt{Spring Framework}. Opiera się ono na zbiorze adnotacji walidujących, które umieszcza się nad polami klas. Można wykorzystać gotowe adnotacje lub wyrażenia regularne. Przykład walidacji danych został przedstawiony na listingu \ref{lst:validation}.

\begin{listing}[H]
    \begin{minted}[linenos, breaklines]{java}
@DateTimeFormat(pattern = "yyyy-MM-dd")
private LocalDate dateOfBirth;
\end{minted}
    \caption{Przykład walidacji danych przy pomocy adnotacji w Spring Framework}
    \label{lst:validation}
\end{listing}

\section{Testy}

System został przetestowany aby upewnić się, że działa on poprawnie. Podczas prac zrezygnowano z testów jednostkowych, ponieważ ich wyniki nie byłyby miarodajne. Możliwe byłoby przetestowanie jedynie warstwy serwisowej która nie realizuje pełnej logiki biznesowej - część z niej jest realizowana poprzez repozytoria. W związku z tym zdecydowano się na przeprowadzenie jednie testów akceptacyjnych w symulowanym środowisku produkcyjnym. Dzięki nim stwierdzono, że aplikacja działa poprawnie i spełnia założenia.